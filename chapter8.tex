\chapter{Future Work}

All though the Utsida prototype was developed to a degree needed to complete this research project, the prototype could be extended with new features and future research projects could focus on different approaches of evaluation. This chapter details possible new features for Utsida, possible improvements to the CBR-RS, and ideas for further research projects.

\section{Web Application Features}

Several suggestions for improving or expanding Utsida emerged throughout the development process. Interviews with exchange course advisers, feedback from usability testing on both students and exchange course advisers, and textual answers from the questionnaires all proposed interesting and meaningful features which could be included in the prototype. Some of the features were implemented, but many were left out because they were outside the scope of this research project. The following list describes the most noteworthy suggestions for the web application:

\begin{description}
    \item[Let advisers make comments on course match applications:] Currently, a course match in Utsida is either approved or disapproved. These matches often included premises to be approved, or they may only be valid for a certain group of students. By implementing a comment functionality for advisers, they could denote this information, making it more useful for students reading it.
    \item[Automatically expiration time on approval of course matches:] Course matches tend to be irrelevant if they are too old, due to changes in the content the courses offers. It could therefore be useful to automatically review or flag old courses.
    \item[Implement filters on the course matches table:] The table of course matches currently includes a search bar, but no filtering. One suggestion from students was to filter the table based on institute or faculty. This could make it easier to find relevant course matches on the site.
    \item[Support for multiple departments:] Include support for users to have multiple belonging departments. This is due to some programmes focusing on several subjects in multiple departments.
    \item[Interactive map of exchange occurrence:] The map on the index page of the web application only display the number of students who have been on exchange in a specific country. In the feedback from usability tests, it was discovered that many students wanted to be able to click on a specific country on the map to get recommendations.
\end{description}

\section{Expanding and Improving the Exchange Experience Concept}

Several attributes that were mentioned in the feedback from the usability tests and questionnaires were not included in the exchange experience concept. These suggestions included cost of living, weather and climate, tuition fees, university rankings, and whether a trip was arranged through a specific program such as Erasmus.\footnote{The European Nation's program for exchange.} None of these are possible to extract from the experience reports, so the data would have to come from external APIs. However, if good sources for such data is found, adding more attributes to the concept would be beneficial for the CBR-RS to recommend more personalized and relevant exchange experiences. For example, Numbeo\footnote{https://www.numbeo.com/cost-of-living/} offers a universal API for cost of living in all countries and cities, and could be used to add cost of living for the country in each experience. 

\section{Adaption in the CBR-RS}

Systems utilizing CBR often involves some form of learning and adaption, which is done in the 3. and 4. step in the CBR-cycle. This is also possible in the Utsida prototype, but was considered out of the scope of this project. To continually expand the case-base, Utsida could automatically parse and add new experience reports as cases in the system as they are published on OIR's site. To enable this functionality, Utsida would have to be configured to accept the format of the new experience reports system as of 2017. To learn more about the preferences of users, Utsida could require feedback on the recommendations, and then adjust the similarity score of similar suggestions for further usage.

\section{Research}

Due to lack of time and resources such as money and participants, Utsida was not evaluated in real-life conditions. The following suggestions could be possible ways to study Utsida's effect in real-life.

\subsection{Testing Utsida in an application Process}

By continuing working on Utsida, the prototype could be used in the real application process for a collection of students. Data could then be collected to find out what these students thought of the application process and compare it to the students who went through the standard application process. The resulting data could be used to further argue that Utsida should be included in the OIR's process, and further document its effect.

\subsection{Observing Real Effects of Utsida}

Utsida could be further evaluated by including the system fully in the application process for an exchange program at NTNU. It would then be of value to compare the number of students applying before and after the inclusion of Utsida. If the number of students applying for an exchange program increased significantly it could be argued that a system like Utsida has a definitive effect on the number of students that go on an exchange program. 



\cleardoublepage