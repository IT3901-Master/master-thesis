%===================================== CHAP 5 =================================

\chapter{Future Work}

All though the system was completed to a degree needed to complete this study; more can be done both regarding continuing the development of the system and conducting research with it. this chapter includes features which would be meaningful to implement in the system, improvements which could be made to the CBRS, and ideas for doing further research with Utsida.

\section{System Features}
Suggestions for improving the system emerged throughout the development process. Interviews with exchange course advisers, feedback from usability testing on both students and exchange course advisers, and textual answers from the questionnaire all proposed interesting and meaningful features which could be included in the system. Some of these features were implemented, but many were left out because they were out of the scope for this research. The following list describes the most noteworthy suggestions which would be meaningful to implement in the system.

\begin{description}
    \item[Let advisers make comments on course match applications:] Currently, a course match in Utsida is either approved or disapproved. These matches often included premises to be approved, or they may only be valid for a certain group of students. By implementing a comment functionality for advisers, they could denote this information, making it more useful for students reading it.
    \item[Automatically expire course matches which are too old:] Course matches tend to be irrelevant if they are too old, due to changes at universities. There could be an automatic review of the course matches which could, for example, expire old ones.
    \item[Implement filters on the course matches table:] The course matches table currently includes a search bar, but no filtering. Some students wanted to filter these matches further for example on institute or faculty. This would make it easier to find relevant course matches on the site.
    \item[Support for multiple departments:] Include support for users to have multiple belonging departments. This is due to some studies focusing on several subjects in multiple departments.
\end{description}

\section{Expanding and Improving the Model of the CBRS}
The CBR part of the system could also be expanded with several possible features and improvement that emerged during the development process. Also, several sources of data used by the system are now deprecated, meaning new up to date sources should be used, and adapted to the system. The following list describes the suggested improvements.

\begin{description}
    \item[More factor/attributes:]  The CBRS can be extended to have options for more motivational factors in the query. Suggestions include cost of living, weather and climate, tuition fees,  university rankings, and whether a trip was arranged through a specific program such as Erasmus\footnote{The European Nation's program for exchange.}.
    \item[Adapting to new experience reports:] A new system for exchange experience reports were released by the OIR in early 2017. Therefore, to support the new experience reports from 2017 onwards, the CBR model has to be configured for the new data. 
    \item[Gain feedback on recommendation relevance:] Currently only the first step of CBR-cycle is used in the CBRS. This could be extended to include the revise step and gain feedback from users on recommendation relevancy.
    \item[Search with courses:] A highly beneficial expansion of the CBR model could be to include courses in the query. The new experience reports also have extended inputs for what courses the student took, making it possible to create more relevant similarity measures. 
\end{description}


\section{Research}
Due to lack of time and resources such as money and participants, there was no extensive research done to test Utsida in real-life conditions. The following suggestions include ways to see Utsida's effect in real-life.

\subsection{Testing Utsida in an application Process}
By continuing working on Utsida, a working prototype could be used in the application process for a collection of students. Data could then be collected to find out what these students thought of the application process, in comparison to the students who went through the standard application process. This data could be used to argue the inclusion of the system in the OIR's  process, and further document its effect.


\subsection{Observing Real Effects of Utsida}
If Utsida was to be included in the application process for an exchange program by the OIR, it would have been interesting to study the number of students applying with the system in use compared to current numbers. If there had been a significant increase of students applying after Utsida was included, it would suggest that the system helped NTNU with its goal of sending more of its students on exchange programs.



\cleardoublepage