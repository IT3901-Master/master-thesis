\chapter{Conclusion}
This chapter concludes the project and presents possible future work on the subject. 

\section{Conclusion}
This research project has found that a system for recommending courses and universities would positively influence the motivation of students for going on an exchange program and could consequently lead to more students participating in the programs. The influence claim is based on the highly positive user feedback received in the evaluation of Utsida, with an average agreement on statements regarding the motivation of 85.8\% (confidence level: 95\%, 0margin of error: 15.5\%). While this margin of error is relatively high due to the less than optimal sample size, the results of this project still indicate a beneficial effect.

Furthermore, CBR is found to be suitable for giving recommendations on universities and courses; the CBR-RS gave relevant recommendations for most (36 out of 40) of the user study participants, and performed significantly better than standard exact match search. While the recommendations alone can not fully be used as an independent solution, they might give students enough inspiration and motivation to initiate the process. If applied in a real setting Utsida would also, with its simplification of the course application process, likely reduce the time used for both students and advisers when approving courses for exchange programs. The simplified and faster application process would lead to more available time for advisers to help students in actual need of help and less time used on students that only need to complete the approval process.

Based on the results that show a high approval of Utsida, it is highly recommended that NTNU invests further in digitizing the whole- or part of the application process for exchange programs. An investment could help NTNU reach their goal of 40\% of the students participating in exchange programs and contribute to the overall goal in Norway of 20\% student mobility. This project has also shown that exchange reports submitted by students can be used in other useful ways. Therefore the data from the experience reports should be public to use (e.g. as an API), and formatted appropriately so that students and others can utilize them in ways that may give a benefit to NTNU. The contributions of this project are not only beneficial to NTNU alone, but could also be used as an inspiration for similar projects at other educational institutions.     


\section{Future Work}

Even though the Utsida prototype was developed to a degree needed to answer this research project, it could be extended with new features, and future research could focus on different approaches of evaluation. This section proposes new features in Utsida, possible improvements to the CBR-RS, and ideas for further research.

\subsection{Web Application Features}

Several suggestions for improving or expanding features of Utsida emerged throughout the development process by reviewing: interviews with exchange course advisers, feedback from usability testing, and textual answers from the questionnaires. Many of the features were outside the scope of this research project and were not implemented; the following list describe the most noteworthy ones:

\begin{description} 
    \item[Let advisers make comments on course approval applications:] Currently, a course application in Utsida is only either approved or disapproved with no comments from advisers. The advisers often wants to include special premises for courses to be approved. By implementing a comment functionality for advisers, advisers could give premises as comments on applications, and through these comments clarify what the student should do to to get the courses approved.
    \item[Expiration on approval of course matches:] Course matches tend to be irrelevant if they are too old, due to changes in the content of the courses. It could therefore be useful to automatically flag old approvals.
    \item[Implement filters on the course match list:] The list of course matches currently only have search functionality. The list functionality could be expanded to include filtering on institute or faculty, making it easier to find relevant course matches.
    \item[Support for multiple departments:] Some study programs focus on many fields of study, and therefore belongs to multiple departments. Support for multiple departments could be implemented in Utsida by letting users register with multiple departments. The CBR-RS could for the recommendations give partial similarity in experiences with either of the user's departments.
    \item[Interactive map of exchange occurrences:] The map on the home page of the web application only display the number of students who have been on exchange in a specific country. This map could be extended to be more interactive, for example by adding support of selecting a specific country to get recommendations.
\end{description}


\subsection{Possible Extensions to the CBR-RS}
The CBR-RS could be further improved by adding more relevant attributes to the concept and by including adaption and learning techniques to personalize and further increase the relevancy of recommendations.

\paragraph{Expanding and Improving the Exchange Experience Concept} Several attributes that were mentioned in the feedback from the usability tests and questionnaires were not included in the exchange experience concept. These suggestions included: cost of living, weather and climate, tuition fees, university rankings, and whether a trip was arranged through a specific program such as Erasmus\footnote{Erasmus: a European Union student exchange program}. None of these are possible to extract from the experience reports, so the data would have to come from external APIs. However, if good sources for such data is found, adding more attributes to the concept would be beneficial for the CBR-RS to recommend more personalized and relevant exchange experiences. For example, Numbeo\footnote{Numbeo: \url{https://www.numbeo.com/cost-of-living/}} offers a universal API for cost of living in all countries and cities, and could be used to add cost of living for the country in each experience. 

\paragraph{Adaption in the CBR-RS} Utsida currently has no form of learning or adaption. Utsida could be extended to learn the preferences of a specific user by requiring feedback on recommendations or by asking for more user information. It could then use this data to increase the relevancy of the recommendations given to the user and adapt the results specifically to the user. For example, if a specific recommendation is a good match for a user, but it only includes a few courses, it could merge its courses with the courses in other similar recommendations. The recommendation would then become more complete. 

\paragraph{Automatic Parsing of New Exchange Experiences} To automatically expand the case-base in Utsida, it would have to parse and add new experience reports as they are published on OIR's site. This could be beneficial to ensure Utsida always has the latest and most relevant data. However, to enable this functionality, Utsida would have to be configured to accept the format of the new experience reports system released in 2017.  

\subsection{Further Research}

If Utsida is further developed to work seamlessly in a production environment, it can be possible to evaluate the system in a real-life situation. The following suggestions could be possible ways to study how Utsida affects motivation in real-life.

\paragraph{Testing Utsida in an Application Process} Utsida could be further evaluated by testing it in a real application process for a collection of students. Data could then be collected to find out what these students thought of using Utsida in the application process and compare it to the students who went through the standard process. The resulting data could be used to further argue that Utsida should be included in the OIR's process, and further document its effect on motivation.

\paragraph{Observing Real Effects of Utsida} Utsida could be further evaluated by including the system fully in the application process for an exchange program at NTNU. It would then be of value to compare the number of students applying before and after the inclusion of Utsida. If the number of students applying for an exchange program increased significantly it could be argued that a system like Utsida has a definitive effect on the number of students that go on an exchange program. 


\cleardoublepage