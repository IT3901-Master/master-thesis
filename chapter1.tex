%===================================== CHAP 1 =================================

\chapter{Introduction}

\section{Motivation}

Studying abroad can be an important way to improve intercultural communication skills that are increasingly more important in international businesses and for cooperation between different cultures.\cite{williams2005exploring} A multicultural experience is also shown to positively affect creative performance and creativity-supporting
cognitive processes\cite{leung2008multicultural} and improve the career opportunities for students\cite{brandenburg2014erasmus}. Many students choose to go abroad by doing a student exchange program or a study-abroad program. These programs usually lasts for 6-12 months where the student is studying at a university in a different country. To apply for a student exchange- or study abroad program at the Norwegian University of Science and Technology (NTNU), the students have to find and pre-approve courses which they desire to take at the university abroad. The process of planning and choosing courses is a time consuming and tedious task, where both the supervisor and student has to undergo a large amount of manual work.

The purpose of this research, is to propose the hypothesis that significantly simplifying the process of choosing exchange courses and getting them approved, as well as making all required information easily available, would increase the interest and motivation among students to study abroad. This process can be replaced by an information system (IS) that automates a large part of the manual work, as well as intelligently suggesting subjects and universities for students by using an intelligent recommendation system with the artificial intelligence (AI) methodology Cased-based Reasoning (CBR).

\section{Current approach}
The current system for choosing and approving exchange courses is mostly done manually at NTNU. The process is required both before and after an exchange or study abroad period. Course selection also have to be pre-approved before applying for a exchange period.

Students are first required to find the university they want to go on exchange to. The possible options are nearly limitless with only some guidance and information on universities with exchange deals that might make the choice easier for the student. After the university is chosen, the student has to find a possible course combination that can be approved as a replacement for the courses originally planned at NTNU. Most of the institutes has no system for either submitting or helping the students find the possible replacements and course matches. The exception is the Department of Information and Computer Science, which utilize a digital table for entering courses. While this table in many ways make it easier for the students to select their university and courses, the data entry does not handle many useful features and data validations. Maintaining a manual table is also tedious work. 

During the exchange period the students often has to change their courses due to unforeseen circumstances making the original approval not valid. To gain a new approval the student generally has to send an email to the adviser of their institute to get the new courses approved. If this process take some time the student might have to choose a course without approval due to the final course selection due date at some universities. 

\section{Previous Research}

A study by Mazzarol and Soutar\cite{mazzarol2002push} shows that students' general motivation is influenced by the amount of information on an university and its courses. Among the several factors which was reviewed in the study, the \enquote{knowledge and awareness} factor proves to be the most influencing one for choosing an international study location. Therefore, when studying how an IS can improve this process, proper information is key. 

The NTNU Office of international relationships conducted a study on 464 of their exchange students in 2016 to find out what factors could help increase the motivation for the students studying abroad. Among the 464 participating students 84\% answered that a list of previously approved exchange courses could improve the number of students that go on exchange and 80\% answered that pre-approved course packages could improve the numbers. This shows the importance and possible usefulness of an information system to be part of the process. 

Several approaches has been made to replace the manual process of choosing courses with an IS. One example is by using a decision support system that advises the students on their course selection based on their program requirements and the course's prerequisites, as done at the University of Dhaka\cite{roushan2014university}. Student course recommendation can be done intelligently by using recommendation engines and data mining techniques to give relevant results. Sherpa\cite{bramucci2012sherpa} is a system that has been made especially for the goal of giving course recommendations. Data mining techniques can also be used to predict whether a student will fail or pass on a course\cite{vialardi2009recommendation}.

CBR is today both a recognized and well-established method in several fields of sciences; such as health science.\cite{begum2011case}. Because CBR is proven to be a suitable methodology for complicated problems where uncertainty is involved \cite{richter2013case}, it has been applied to problems such as diagnosing chronically diseases in combination with data mining \cite{huang2007integrating}, and finding the most suitable study program at a senior high school \cite{mulyana2015case}. Case-Based Recommender Systems, which is discussed in section \ref{sec:case_based_recommender_systems} are also commonly used in decision problems, typically when there's not one true answer, but rather several different good answers. For instance such a system was developed to find the most suitable Massive Open Online Courses (MOOCs) for e-learning. \cite{bousbahi2015mooc}. 

Among the many applications utilizing the CBR methodology, none has covered the issue of finding the most suitable country, university and courses to choose when going on an exchange study. Therefore, based on the motivation for this research, and similar research which has been done, this research will answer the following research questions:

\section{Research Questions}

\begin{itemize}
    \item \textbf{RQ1:} How suitable is the Case-Based Reasoning methodology for the domain of finding a relevant university and subjects for a study abroad- or study exchange program?
    \item \textbf{RQ2:} To what extent can a decision support system for selecting courses improve students' motivation for doing a study abroad- or study exchange program?
\end{itemize}


\section{Methodology}
To answer these research questions, a combination of two research strategies will be used; namely the Design and Creation- and Survey strategy. The Design and Creation strategy will be used as a guidance for creating both a web application, and a Cased-Based Reasoning System. These systems will in turn be used as tools in a survey which will attempt to produce the data needed to answer both RQ1 and partly RQ2. In order to quantify the findings regarding RQ1, a designated test will be performed on the CBR-system to analyze whether the system yields proper recommendations. See chapter \ref{chap:3} for a full description of the research method.

\iffalse
\section{Contributions}

This research's contribution will yield a new computer based artefact; specifically an IS with the goal of both motivating and making it easier for students at NTNU to apply for a study exchange program.

As implied by some of the similar systems and solutions already mentioned, this research will invent a new innovating system that could be extended to other universities. Aside from simplifying this process, the research will contribute by giving new insight and knowledge about what an IS can do to increase the motivation for students to apply for a study exchange program, and if it is possible to completely automate part of this process, as well as providing automated intelligent educated guesses for the most suitable and relevant courses a student should choose. Considering this research objective is primarily designed as a contribution to NTNU, who currently does most of this process manually, if this research and product proves to be successful, it will be a great contribution for NTNU. All though there exists similar systems in other universities which focus on some of the same objectives as this research, commercial-off-the-shelf (COTS) products can not be implemented, when the dependencies and infrastructure is so different for different universities.
\fi

\section{Chapter Descriptions}
This section describes the different chapters of this thesis. The main focus and content will be detailed to gain a brief overview of the thesis structure. 

\subsection*{Chapter 2: Theory}
This chapter presents theory and information about concepts and tools which is beneficial to understand for this research. The main focus lies on CBR, recommender systems and acknowledged software tools for these methodologies. 

\subsection*{Chapter 3: Research Method}
This chapter defines the research methodology used in the research, including the utilized research strategies, data collection methods and how the data was analyzed. 
\subsection*{Chapter 4: Approach}
This chapter elaborates on how the system Utsida is built. Including reasoning for the design, framework and technology choices, as well as how the CBRS is modelled and implemented.

\subsection*{Chapter 5: Results \& Discussion}
This chapter lists all the results and data acquired with the utilized research method with a comprehensive discussion.

\subsection*{Chapter 6: Conclusion}
This chapter includes a conclusion based on the acquired results and the analysis of these, how the project turned out, and ideas for future work.

\cleardoublepage