%===================================== CHAP 1 =================================

\chapter{Introduction}\label{chap:1}

This research project was initiated to make it easier for students at NTNU to apply for a study abroad- or student exchange program by using a prototype, named Utsida, as a proof of concept to replace parts of the current manual process. This chapter elaborates the motivation and background for that initiation. It also details the research questions and goals for the solution, gives a brief overview of the research methodology, and describes the structure of the thesis.

\section{Motivation and Background}
Studying abroad can be an important way to improve intercultural communication skills that are increasingly more important in international businesses and for cooperation between different cultures \cite{williams2005exploring}. A multicultural experience is also shown to enhance creativity \cite{leung2008multicultural} and improve the career opportunities for students \cite{brandenburg2014erasmus}. Because of the beneficial effects of multicultural experiences, The Norwegian University of Science and Technology (NTNU) has set the following goal of international mobility for degree students: \enquote{By 2017, at least 40\% of NTNU's degree students should have a period of study at an educational institution in a foreign country lasting one or two semesters.}\footnote{http://www.ntnu.edu/international-action-plan}. Norway has also committed to the Bologna process \footnote{https://www.ehea.info/}, with one of the goals being that 20\% of the students completing a study in Europe should undertake a minimum of three months of their education in a different country by 2020.

With only 25\% student mobility at NTNU in 2015 \cite{studentutveksling_andel}, NTNU has to, to reach their goal of 40\% international mobility, look at ways to increase motivation for students to do a study abroad- or student exchange program. One of the ways could be to digitalize the current process of applying for a study abroad- or student exchange program; this could increase student motivation and reduce the workload for advisers. The difference between an exchange program and study abroad program is that a student exchange program involves students exchanging places and the student only pays tuition fees at their home university. In a study abroad program, however, the student is enrolled in the abroad university and pay the tuition fee. Both of these programs have similar application procedures at NTNU, and will, henceforth, be referred to as an exchange program. Before applying for an exchange program at NTNU, students have to find and approve courses to take at the university abroad. This process of approving and finding courses is mostly done manually by both advisers and students. It could be replaced by an information system (IS) that simplifies the process and consequently increase students' motivations for exchange. 

\section{Personal Motivation}
Replacing tedious manual work with an intuitive and easy to use IS is something that motivates both of the authors. The project was first proposed by our supervisor, Associate Prof. Rune Sætre, as a study exploring the similarity between courses from different universities. However, due to several ongoing similar studies, we instead proposed to focus on finding ways to improve the exchange course approval process. One of the authors, Truls Mørk Pettersen, has previously done an exchange program where he experienced that the current process has room for improvement. He therefore wanted to study ways to increase the motivation and reduce the workload for students.

\section{Research Questions and Goals}\label{RQ}
The primary goal of this project is to create a prototype, named Utsida, that increase the motivation for students at NTNU to apply for an exchange program. To achieve this goal the prototype should simplify the exchange course approval process and give relevant recommendations on courses and universities. The second goal is to use the popular case-based reasoning (CBR) methodology for producing the recommendations. The following list summarizes the research goals.

\begin{itemize}[noitemsep]
    \item Goal 1: Create a prototype that improves the motivation for students at NTNU to apply for an exchange program.
    \item Goal 2: Use the CBR methodology to give relevant recommendations on exchange program universities and courses.
\end{itemize}

Based on these goals, and the knowledge gained in the preliminary research (Ch. \ref{chap:2}) the specific research questions this study seeks to answer are:

\begin{itemize}
    \item \textbf{RQ1:} What effect would an information system for recommending and assisting exchange course selection have on students' motivation for doing a study abroad- or student exchange program at NTNU?
    \item \textbf{RQ2:} How suitable is the case-based reasoning methodology for recommending relevant universities and courses for a study abroad- or student exchange program?
\end{itemize}


\section{Methodology}
Oates' \cite{oates2005researching} model of the research process, detailed in Chapter \ref{chap:4}, was used to define the chosen research methodology. The research questions were developed from motivation (Ch. \ref{chap:1}), experiences (Ch. \ref{chap:2}) and related work (sec. \ref{related_work}). The Design and Creation research strategy was used to create Utsida, a prototype system with two parts: a web application and a cased-based reasoning recommender system (CBR-RS). These systems were, in turn, evaluated through questionnaires which produced the data needed to answer RQ1 and partly RQ2. To fully answer RQ2, and further quantify the result of the questionnaires, an offline experiment was performed on the CBR-RS to analyze whether it yielded relevant recommendations. See Chapter \ref{chap:4} for a full description of the research methods used.
    

\section{Thesis Structure}
The thesis is structured in eight chapters. Firstly, Chapter 1 and 2 answers why this research was conducted by exploring the motivation, problem context, and related work. Chapter 3 then introduces theory on essential concepts identified in the preliminary research. Furthermore, Chapter 4 and 5 answers how the research project was conducted by detailing the chosen research methods and the implementation of the prototype. Chapter 6 shows the results produced by the research methods, while chapter 7 interprets these results in regards to the research questions and presents a conclusion. Finally, Chapter 8 ends the thesis by presenting ideas for future work. 

The following paragraphs further detail the content and purpose of each chapter.

\paragraph{Chapter 2: Preliminary Research}
Describes the preliminary research that led to the production of the research questions and the goals of this project. This includes knowledge attained on the problem domain, review of the current approach, and a summary of related work found through literature review.

\paragraph{Chapter 3: Theory}
Presents the theory and information about concepts and tools which were essential for creating the prototype needed to answer the research questions. The main focus lies on CBR and recommender systems. 

\paragraph{Chapter 4: Research Method}
Describes the methodological choices of this thesis. This includes presenting the overarching research strategy: Design and Creation, used as a guideline to develop the prototype. Furthermore, the chapter details the data collection methods and data analysis used to evaluate the prototype and consequently answer the research questions. These methods are questionnaires, offline experiments, and quantitative data analysis. Finally, important ethical and practical issues that influence the project are elaborated. 

\paragraph{Chapter 5: Implementation}
Elaborates on the architecture and implementation of the prototype, Utsida. Including decisions made for design, frameworks, and technologies, as well as how the CBR-RS was modeled and implemented.

\paragraph{Chapter 6: Results}
Presents the key results acquired with the utilized research methods on the prototype. These results are represented by diagrams and tables.

\paragraph{Chapter 7: Discussion and Conclusion}
Includes a comprehensive discussion were the results are interpreted in regards to the research questions, and possible limitations of the research project are presented. Ends by concluding the thesis based on the interpreted results.

\paragraph{Chapter 8: Future Work}
Presents ideas for future work on the subject. This includes new features for the prototype, extensions to the concept in the CBR-RS, and ways to evaluate the system in real-life conditions. 


\cleardoublepage