\usepackage{setspace}
\usepackage{graphicx}
\usepackage{amssymb}
\usepackage{mathrsfs}
\usepackage{amsthm}
\usepackage{amsmath}
\usepackage{color}
\usepackage[utf8]{inputenc}
\usepackage[Lenny]{fncychap}
\usepackage[pdftex,bookmarks=true]{hyperref}
\usepackage[pdftex]{hyperref}
\usepackage[table,xcdraw]{xcolor}
\usepackage[font=small,labelfont=bf]{caption}
\usepackage{fancyhdr}
\usepackage{times}
\usepackage[numbers]{natbib}
\usepackage{float}
\usepackage[parfill]{parskip}
\usepackage{enumitem}
\usepackage{csquotes}
\usepackage{tabulary}
\usepackage{appendix}
\usepackage{subcaption}
\usepackage{tikz}
\usepackage{placeins}
\usepackage{pgfplots}
\usepackage{calc}
\usepackage{ifthen}
\usepackage{pgf-pie}
\usepackage{listings}
\usepackage{minted}
\usepackage{multirow}
\usepackage{url}

\restylefloat{figure}
\setcitestyle{square}
\captionsetup{format=hang}
\pgfplotsset{compat=1.13}

\hypersetup{
    colorlinks,%
    citecolor=black,%
    filecolor=black,%
    linkcolor=black,%
    urlcolor=black
}

\pgfplotsset{ every non boxed x axis/.append style={x axis line style=-},
     every non boxed y axis/.append style={y axis line style=-}}
     
\usetikzlibrary{arrows,shapes,positioning,shadows,trees}

\newcommand{\slice}[4]{
  \pgfmathparse{0.5*#1+0.5*#2}
  \let\midangle\pgfmathresult

  % slice
  \draw[thick,fill=black!10] (0,0) -- (#1:1) arc (#1:#2:1) -- cycle;

  % outer label
  \node[label=\midangle:#4] at (\midangle:1) {};

  % inner label
  \pgfmathparse{min((#2-#1-10)/110*(-0.3),0)}
  \let\temp\pgfmathresult
  \pgfmathparse{max(\temp,-0.5) + 0.8}
  \let\innerpos\pgfmathresult
  \node at (\midangle:\innerpos) {#3};
}

%----------------Defining colors-----------
\definecolor{blueMariusLight}{RGB}{72, 188, 255}
\definecolor{corsaneOrange}{RGB}{255, 177, 72}
\definecolor{corsanenGrey}{RGB}{217, 217, 217}
\definecolor{prettyGreen}{RGB}{92, 184, 92}
\definecolor{bootstrapBlue}{RGB}{66,139,202}


%-------------Main style for WBS figures----------------
\tikzset{
  basic/.style  = {draw=none, text width=2cm, font=\sffamily, rectangle},
  root/.style   = {basic, rounded corners=1pt, align=center},
  level 2/.style = {basic, rounded corners=1pt, thin,align=center, fill=bootstrapBlue,text=white,
                   text width=7.5em},
  level 3/.style = {basic, thin, align=left, rounded corners=1pt, fill=corsanenGrey, text width=6.5em}
}

\newcommand{\HRule}{\rule{\linewidth}{0.5mm}}

\renewcommand*\contentsname{Table of Contents}

\pagestyle{fancy}
\fancyhf{}
\renewcommand{\chaptermark}[1]{\markboth{\chaptername\ \thechapter.\ #1}{}}
\renewcommand{\sectionmark}[1]{\markright{\thesection\ #1}}
\renewcommand{\headrulewidth}{0.1ex}
\renewcommand{\footrulewidth}{0.1ex}
\fancypagestyle{plain}{\fancyhf{}\fancyfoot[LE,RO]{\thepage}\renewcommand{\headrulewidth}{0ex}}

\linespread{1.5}