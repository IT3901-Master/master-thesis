

\pagestyle{fancy}
\fancyhf{}
\renewcommand{\chaptermark}[1]{\markboth{\chaptername\ \thechapter.\ #1}{}}
\renewcommand{\sectionmark}[1]{\markright{\thesection\ #1}}
\renewcommand{\headrulewidth}{0.1ex}
\renewcommand{\footrulewidth}{0.1ex}
\fancyfoot[LE,RO]{\thepage}
\fancypagestyle{plain}{\fancyhf{}\fancyfoot[LE,RO]{\thepage}\renewcommand{\headrulewidth}{0ex}}


\begin{center}
\section*{\Huge\textit{Sammendrag}}
\end{center}



%\addcontentsline{toc}{chapter}{Summary}	
$\\[0.5cm]$

NTNU har mål om å være internasjonalt fremragende. Et av initiativene til å oppnå målet er å øke antallet studenter som drar på utveksling. Denne avhandlingen introduserer en prototype laget for å forbedre studenters motivasjon for utveksling ved å anbefale fag og universiteter, og forenkle prosessen med å forhåndsgodkjenne fag. Prototypen, kalt \textit{Utsida}, bruker case-based reasoning og erfaringsrapportene til tidligere utveklsingsstudenter for å gi anbefalinger. \enquote{Design and Creation} forskningsstrategien ble brukt til å utvikle og evaluere de to delene av Utsida. Den ene delen er en webapplikasjon som håndterer brukerinteraksjon og datalagring, og den andre delen er et case-based reasoning recommender system som produserer anbefalingene. Utsida-prototypen ble evaluert ved å bruke to metoder. Den første metoden var en brukerstudie hvor studenter ved NTNU kunne teste Utsida og svare på et medfølgende spørreskjema. Den andre metoden testet case-based reasoning recommender system-delen i et offline eksperiment med simulerte brukerforespørsler. Brukerstudien viste at Utsida har en svært positiv effekt på studenters motivasjon for å søke utveksling. Flertallet av studentene mottok også relevante anbefalinger for både universiteter og fag. De signifikante resultatene fra offline eksperimentet viste en høy relevanse på den implementerte anbefalingsmetoden og støttet resultatene av brukerstudien.


\clearpage