\clearpage
\pagenumbering{roman} 				
\setcounter{page}{1}

\pagestyle{fancy}
\fancyhf{}
\renewcommand{\chaptermark}[1]{\markboth{\chaptername\ \thechapter.\ #1}{}}
\renewcommand{\sectionmark}[1]{\markright{\thesection\ #1}}
\renewcommand{\headrulewidth}{0.1ex}
\renewcommand{\footrulewidth}{0.1ex}
\fancyfoot[LE,RO]{\thepage}
\fancypagestyle{plain}{\fancyhf{}\fancyfoot[LE,RO]{\thepage}\renewcommand{\headrulewidth}{0ex}}


\pagestyle{empty}
\begin{center}
\section*{\Huge\textit{Sammendrag}}
\end{center}



%\addcontentsline{toc}{chapter}{Summary}	
$\\[0.5cm]$

Denne oppgaven er et resultat av forskning med målet å undersøke potensialet til Case-Based Reasoning for å anbefale relevante universiteter og fag for studenter som vil dra på utveklsing, og hvordan et informasjonssystem som bruker denne teknologien kan påvirke studenters motivasjon for å ville dra på utveskling. Ved å utvikle et system på en iterativ måte med brukerene i fokus, bruke erfaringer fra tidligere utveklsingsstudetner og å tilpasse et Case-Based Recommender System for dette domenet, streber dette prosjektet etter å gi gode anbefalinger for studenter, og å øke deres motivasjon ved å gjøre søknadesprosessen for utveklsing enklere. Resultatene ble samlet ved å få et utvalg studenter på NTNU som er interessert i utveksling, eller har en tilknytning til det til å teste det utviklede systemet og svare på en tilhørende spørreundersøkelse. Disse resultatene indikerer at om det utviklede systemet var i bruk, ville flere studenter dratt på utveksling, og at flertalle av studenter får anbefalt både relevant universiteter og fag gjennom systemet.


\clearpage