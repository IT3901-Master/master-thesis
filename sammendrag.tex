\clearpage
\pagenumbering{roman} 				
\setcounter{page}{1}

\pagestyle{fancy}
\fancyhf{}
\renewcommand{\chaptermark}[1]{\markboth{\chaptername\ \thechapter.\ #1}{}}
\renewcommand{\sectionmark}[1]{\markright{\thesection\ #1}}
\renewcommand{\headrulewidth}{0.1ex}
\renewcommand{\footrulewidth}{0.1ex}
\fancyfoot[LE,RO]{\thepage}
\fancypagestyle{plain}{\fancyhf{}\fancyfoot[LE,RO]{\thepage}\renewcommand{\headrulewidth}{0ex}}


\pagestyle{empty}
\begin{center}
\section*{\Huge\textit{Sammendrag}}
\end{center}



%\addcontentsline{toc}{chapter}{Summary}	
$\\[0.5cm]$

NTNU har mål om å være internasjonalt fremragende. Et av initiativene er å øke den internasjonale mobiliteten til NTNUs gradsstudenter. Debbe økningen kan bli oppnådd ved å motivere flere studenter til å dra på utveklsing. Denne avhandlingen introduserer en prototype laget for å forbedre studenters motivasjon for utveksling ved å anbefale fag og universiteter, og forenkle prosessen med å forhåndsgodkjenne fag. Prototypen, kalt \textit{Utsida}, bruker case-based reasoning (CBR) og erfaringsrapportene til tidligere utveklsingsstudenter for å gi anbefalinger basert på erfaringer fra tidligere utvekslingsstudenter. Utviklingen av prototypen ble gjennomført iterativt gjennom \enquote{Design and Creation} forskningsstrategien, og utformet med brukersentrert design. Utsida-prototypen består av to deler, hvor hver del er rettet mot hvert sitt forskningsspørsmål. Den ene delen er en webapplikasjon som håndterer brukerinteraksjon og datalagring, og den andre delen er et case-based reasoning recommender System (CBR-RS) som produserer anbefalingene. Utsida-prototypen ble evaluert ved å bruke to metoder; Den første er en brukerstudie hvor studenter ved NTNU testet prototypen og svarte på et medfølgende spørreskjema. Den andre metoden testet CBRS-delen i et offline eksperiment med simulerte brukerforespørsler. Brukerstudien viste at Utsida har en svært positiv effekt på motivasjon for utveksling med en gjennomsnittlig enighet på 86\% for påstandene om økt motivasjon. Videre fikk også flertallet av studentene relevante anbefalinger for både universiteter og fag. De endelige resultatene fra offline eksperimentet var statistisk signifikante og støttet resultatene av brukerstudien.


\clearpage