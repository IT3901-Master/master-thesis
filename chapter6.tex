%===================================== CHAP 5 =================================

\chapter{Results}\label{ch:results&discussion}

This chapter provides all final results yielded by the evaluations of the prototype, namely the questionnaires and the offline experiment on the CBR-RS.

\section{Questionnaire 1: Motivational Factors}

The result from questionnaire 1 gave valuable insights on students' motivational factors for exchange, and was also used to weight the different attributes in the CBR-RS (sec. \ref{sec:weighting}). The questionnaire had 84 participants of which 52 of them had participated in an exchange program. 67 were students at NTNU and 17 were students at other higher educational institutions in Norway. All of the answers included numerical ratings between 1-7, describing how important the participant thought each of the attributes in the questionnaire was. Additionally, 36 of the replies included an optional textual answer. Table \ref{tab:attribute_ranking} shows the values for each attribute scaled to a range between 1-10, while Appendix \ref{appendix:word_frequency} displays the result of the theme analysis, that is, the most frequent category of factors in the textual answers.

\begin{table}[h]
\small
\captionsetup{width=0.8\textwidth}
\caption[Questionnaire 1 result]{Questionnaire 1, motivational factors result. Ranked by CV. N=84. \\ *SD: Standard Deviation, CV: Coefficient of Variation}
\centering
\label{tab:attribute_ranking}
\begin{tabulary}{\textwidth}{|L|R|R|R|}
\hline
\textbf{Attribute} & \textbf{Mean (1-10)} & \textbf{SD (1-10)} & \textbf{CV (0-1)} \\ \hline \hline
Language & 8.2 & 1.87 & 0.23 \\ \hline
Social quality & 7.02 & 1.75 & 0.25 \\ \hline
Academic quality & 7.11 & 2.12 & 0.30 \\ \hline
Administrative support and reception & 6.5 & 2.27 & 0.35 \\ \hline
Country/Continent & 6.84 & 2.45 & 0.36 \\ \hline
Quality residential & 5.27 & 1.92 & 0.36 \\ \hline
Cost of living & 5.66 & 2.13 & 0.38 \\ \hline
Climate/weather & 6.22 & 2.53 & 0.41 \\ \hline
Availability of residentials & 5.31 & 2.22 & 0.42 \\ \hline
\end{tabulary}
\end{table}

\section{Questionnaire 2: User Study of Utsida}

Questionnaire 2 had 40 participating students from NTNU, where 20 of the replies included an optional textual feedback. 27 (67.5\%) of the participants had not been on an exchange program, while 13 (32.5\%) had. Figure \ref{fig:faculty_demographic} shows the faculty demographic of the participants.

\begin{figure}[h]
    \small
    \centering
    \begin{tikzpicture}[scale=0.7]
        \pie[text=legend,sum = auto, after number = ]{
            8/Natural Sciences (NV), 
            10/Engineering (IV), 
            14/Information Technology and Electrical Engineering (IE), 
            5/Economics and Management (OK),
            2/Social and Educational Sciences (SU),
            1/Humanities (HF)
        }
    \end{tikzpicture}
    \caption{Faculty demographic of participants in Questionnaire 2}
    \label{fig:faculty_demographic}
\end{figure}

The results from the closed questions are presented first, followed by the findings from the open question which is summarized in short. The results from the closed questions are divided into two parts; Part one for results that contribute to answering RQ1 (motivational effect), and part two for those that contribute to RQ2 (suitability of CBR).

\subsection{Part 1: Motivational Effect and Use}

The following figures display results from the questions concerning the motivational effect Utsida may have on students to apply for an exchange program. The degree of agreement ranges from \textit{1: Strongly disagree} to \textit{5: Strongly agree} as a Likert-type question.


\begin{figure}[h]

    \centering
    
    \begin{tikzpicture}[font=\small, trim axis left, trim axis right]
        \begin{axis}[
            height=4.5cm,
            width=9cm,
            ybar,
            bar width=20pt,
            xlabel={Degree of agreement (1-5)},
            ylabel={Number of answers},
            ymin=0,
            ytick={2,4,6,8,10,12,14},
            xtick=data,
            axis x line=bottom,
            axis y line=left,
            enlarge x limits=0.2,
            symbolic x coords={1, 2, 3, 4, 5},
            xticklabel style={anchor=base,yshift=-\baselineskip},
            nodes near coords={\pgfmathprintnumber\pgfplotspointmeta}
        ]
        
          \addplot[fill=bootstrapBlue] coordinates {
            (1, 0)
            (2, 1)
            (3, 3)
            (4, 14)
            (5, 9)
          };
        \end{axis}
    \end{tikzpicture}
    \captionsetup{format=hang}
    \caption[Result for statement 1, questionnaire 2] {\textbf{Statement}: "I think my motivation for exchange would increase if Utsida was in use". \textbf{Group:} Students who have \textbf{not} participated in an exchange program. (N = 27)}
    \label{fig:motivational_results_1}
    
\end{figure}

\begin{figure}[h]

    \centering
    
    \begin{tikzpicture}[font=\small, trim axis left, trim axis right]
        \begin{axis}[
            height=4.5cm,
            width=9cm,
            ybar,
            bar width=20pt,
            xlabel={Degree of agreement (1-5)},
            ylabel={Number of answers},
            ymin=0,
            ytick={2,4,6,8,10,12,14},
            xtick=data,
            axis x line=bottom,
            axis y line=left,
            enlarge x limits=0.2,
            symbolic x coords={1, 2, 3, 4, 5},
            xticklabel style={anchor=base,yshift=-\baselineskip},
            nodes near coords={\pgfmathprintnumber\pgfplotspointmeta}
        ]
        
          \addplot[fill=bootstrapBlue] coordinates {
            (1, 1)
            (2, 2)
            (3, 6)
            (4, 6)
            (5, 12)
          };
        \end{axis}
    \end{tikzpicture}
    
    \caption[Result for statement 2, questionnaire 2]{\textbf{Statement:} "I thought it was easy to use Utsida". \\ \textbf{Group:} Students who have \textbf{not} participated in an exchange program (N = 27)}
    \label{fig:motivational_results_2}
    
\end{figure}

\begin{figure}[h]

    \centering
    
    \begin{tikzpicture}[font=\small, trim axis left, trim axis right]
        \begin{axis}[
            height=4.5cm,
            width=9cm,
            ybar,
            bar width=20pt,
            xlabel={Degree of agreement (1-5)},
            ylabel={Number of answers},
            ymin=0,
            ytick={2,4,6,8,10,12,14},
            xtick=data,
            axis x line=bottom,
            axis y line=left,
            enlarge x limits=0.2,
            symbolic x coords={1, 2, 3, 4, 5},
            xticklabel style={anchor=base,yshift=-\baselineskip},
            nodes near coords={\pgfmathprintnumber\pgfplotspointmeta}
        ]
        
          \addplot[fill=bootstrapBlue] coordinates {
            (1, 0)
            (2, 0)
            (3, 1)
            (4, 3)
            (5, 9)
          };
        \end{axis}
    \end{tikzpicture}
    
    \caption[Result for statement 3, questionnaire 2]{\textbf{Statement:} "I think Utsida would have simplified my application process". \\ \textbf{Group:} Students who have participated in an exchange program (N = 13).}
    \label{fig:motivational_results_3}
    
\end{figure}


\begin{figure}
    \centering
    \begin{subfigure}[b]{0.4\textwidth}
        \begin{tikzpicture}[font=\small, trim axis left, trim axis right]
        \begin{axis}[
            height=5cm,
            width=7cm,
            ybar,
            bar width=20pt,
            xlabel={Degree of agreement (1-5)},
            ylabel={Number of answers},
            ymin=0,
            ytick={2,4,6,8,10,12,14},
            xtick=data,
            axis x line=bottom,
            axis y line=left,
            enlarge x limits=0.2,
            symbolic x coords={1, 2, 3, 4, 5},
            xticklabel style={anchor=base,yshift=-\baselineskip},
            nodes near coords={\pgfmathprintnumber\pgfplotspointmeta}
        ]
        
          \addplot[fill=bootstrapBlue] coordinates {
            (1, 0)
            (2, 0)
            (3, 2)
            (4, 11)
            (5, 14)
          };
        \end{axis}
    \end{tikzpicture}
        \caption{\textbf{Group:} Students who have \textbf{not} participated in an exchange program (N = 13)}
        \label{fig:motivational_results_4_p1}
    \end{subfigure}
    ~ \qquad %add desired spacing between images, e. g. ~, \quad, \qquad, \hfill etc. 
      %(or a blank line to force the subfigure onto a new line)
    \begin{subfigure}[b]{0.4\textwidth}
        \begin{tikzpicture}[font=\small, trim axis left, trim axis right]
        \begin{axis}[
            height=5cm,
            width=7cm,
            ybar,
            bar width=20pt,
            xlabel={Degree of agreement (1-5)},
            ylabel=\empty,
            ymin=0,
            ytick={2,4,6,8,10,12,14},
            xtick=data,
            axis x line=bottom,
            axis y line=left,
            enlarge x limits=0.2,
            symbolic x coords={1, 2, 3, 4, 5},
            xticklabel style={anchor=base,yshift=-\baselineskip},
            nodes near coords={\pgfmathprintnumber\pgfplotspointmeta}
        ]
        
          \addplot[fill=bootstrapBlue] coordinates {
            (1, 0)
            (2, 1)
            (3, 0)
            (4, 8)
            (5, 4)
          };
        \end{axis}
    \end{tikzpicture}
        \caption{\textbf{Group:} Students who \textbf{have} participated in an exchange program (N = 27)}
        \label{fig:motivational_results_4_p2}
    \end{subfigure}
    \caption[Result for statement 4, questionnaire 2]{\textbf{Statement:} "I think Utsida can contribute to an increased number of students who choose to participate in an exchange program"}
    \label{fig:motivational_results_4}
\end{figure}

\FloatBarrier


To make an inference on whether OIR should include Utsida in their exchange program application process, one question asked if the participants would recommend Utsida to the OIR. The results are shown in Figure \ref{fig:office_recommendation} and 39 out of 40 participants would have recommended that OIR included Utsida in the process. 

\FloatBarrier

\begin{figure}[h]
    \centering
    \begin{subfigure}[b]{0.4\textwidth}
        \begin{tikzpicture}[scale=0.7]
            \pie[ sum = auto , after number = ]{27/Yes}
        \end{tikzpicture}    
        
        \caption{Group: Students who have \textbf{not} participated in an exchange program (N = 27)}
        \label{fig:office_recommendation_p1}
    \end{subfigure}
    ~ \qquad %add desired spacing between images, e. g. ~, \quad, \qquad, \hfill etc. 
      %(or a blank line to force the subfigure onto a new line)
    \begin{subfigure}[b]{0.4\textwidth}
        \begin{tikzpicture}[scale=0.7]
            \pie[ sum = auto , after number = ]{12/Yes, 1/No}
        \end{tikzpicture}  
       
        \caption{Group: Students who \textbf{have} participated in an exchange program (N = 13)}
        \label{fig:office_recommendation_p2}
    \end{subfigure}
    \caption[Result from question on recommendation to OIR]{Results from question: "Would you recommend The Office of International Relations to include this system in the exchange application process?"}
    \label{fig:office_recommendation}
\end{figure}


\FloatBarrier

The summary in Table \ref{tab:likert_type_summary} shows the mode and frequencies for all the Likert-type questions. Each frequency also has a percentage value indicating the share of answers on the degree of agreement. In addition, the two target groups of questions has a Cronbach's Alpha value calculated to evaluate their internal consistency.


\FloatBarrier
\begin{table}[h]
\centering
\caption{Summary of results from Likert-type questions}
\label{tab:likert_type_summary}
\resizebox{\textwidth}{!}{%
\begin{tabular}{lcccccc}
 &  & \multicolumn{5}{c}{\textbf{\begin{tabular}[c]{@{}c@{}}Frequencies \\ (degree of agreement)\end{tabular}}} \\ \cline{2-7} 
\multicolumn{1}{c|}{\textbf{Statement}} & \multicolumn{1}{c|}{\cellcolor[HTML]{BBDAFF}\textbf{Mode}} & \multicolumn{1}{c|}{\cellcolor[HTML]{BBDAFF}\textbf{1}} & \multicolumn{1}{c|}{\cellcolor[HTML]{BBDAFF}\textbf{2}} & \multicolumn{1}{c|}{\cellcolor[HTML]{BBDAFF}\textbf{3}} & \multicolumn{1}{c|}{\cellcolor[HTML]{BBDAFF}\textbf{4}} & \multicolumn{1}{c|}{\cellcolor[HTML]{BBDAFF}\textbf{5}} \\ \hline
\multicolumn{7}{|c|}{\cellcolor[HTML]{C0C0C0}Students that \textbf{have not} been on exchange    \hspace{1cm}   (Cronbach's Alpha = 0.916)} \\ \hline
\multicolumn{1}{|l|}{\textit{\begin{tabular}[c]{@{}l@{}}I think my motivation for exchange \\ would increase if Utsida was in use\end{tabular}}} & \multicolumn{1}{c|}{\cellcolor[HTML]{CBCEFB}4} & \multicolumn{1}{c|}{0} & \multicolumn{1}{c|}{\begin{tabular}[c]{@{}c@{}}1\\ (3.7\%)\end{tabular}} & \multicolumn{1}{c|}{\begin{tabular}[c]{@{}c@{}}3\\ (11.1\%)\end{tabular}} & \multicolumn{1}{c|}{\cellcolor[HTML]{FFFFFF}\begin{tabular}[c]{@{}c@{}}14\\ (51.9\%)\end{tabular}} & \multicolumn{1}{c|}{\begin{tabular}[c]{@{}c@{}}9\\ (33.3\%)\end{tabular}} \\ \hline
\multicolumn{1}{|l|}{\textit{I thought it was easy to use Utsida}} & \multicolumn{1}{c|}{\cellcolor[HTML]{CBCEFB}5} & \multicolumn{1}{c|}{\begin{tabular}[c]{@{}c@{}}1\\ (3.7\%)\end{tabular}} & \multicolumn{1}{c|}{\begin{tabular}[c]{@{}c@{}}2\\ (7.4\%)\end{tabular}} & \multicolumn{1}{c|}{\begin{tabular}[c]{@{}c@{}}6\\ (22.2\%)\end{tabular}} & \multicolumn{1}{c|}{\begin{tabular}[c]{@{}c@{}}6\\ (22.2\%)\end{tabular}} & \multicolumn{1}{c|}{\begin{tabular}[c]{@{}c@{}}12\\ (44.4\%)\end{tabular}} \\ \hline
\multicolumn{1}{|l|}{\textit{\begin{tabular}[c]{@{}l@{}}I think Utsida can contribute to an increased \\ number of students who choose to \\ participate in an exchange program\end{tabular}}} & \multicolumn{1}{c|}{\cellcolor[HTML]{CBCEFB}5} & \multicolumn{1}{c|}{0} & \multicolumn{1}{c|}{0} & \multicolumn{1}{c|}{\begin{tabular}[c]{@{}c@{}}2\\ (7.4\%)\end{tabular}} & \multicolumn{1}{c|}{\begin{tabular}[c]{@{}c@{}}11\\ (40.7\%)\end{tabular}} & \multicolumn{1}{c|}{\begin{tabular}[c]{@{}c@{}}14\\ (51.9\%)\end{tabular}} \\ \hline
\multicolumn{7}{|c|}{\cellcolor[HTML]{C0C0C0}Students that \textbf{have been} on exchange    \hspace{1cm}    (Cronbach's Alpha = 0.855)} \\ \hline
\multicolumn{1}{|l|}{\textit{\begin{tabular}[c]{@{}l@{}}I think Utsida would have simplified my \\ application process\end{tabular}}} & \multicolumn{1}{c|}{\cellcolor[HTML]{CBCEFB}5} & \multicolumn{1}{c|}{0} & \multicolumn{1}{c|}{0} & \multicolumn{1}{c|}{\begin{tabular}[c]{@{}c@{}}1\\ (7.7\%)\end{tabular}} & \multicolumn{1}{c|}{\begin{tabular}[c]{@{}c@{}}3\\ (23.1\%)\end{tabular}} & \multicolumn{1}{c|}{\begin{tabular}[c]{@{}c@{}}9\\ (69.2\%)\end{tabular}} \\ \hline
\multicolumn{1}{|l|}{\textit{\begin{tabular}[c]{@{}l@{}}I think Utsida can contribute to an increased \\ number of students who choose to\\ participate in an exchange program\end{tabular}}} & \multicolumn{1}{c|}{\cellcolor[HTML]{CBCEFB}4} & \multicolumn{1}{c|}{0} & \multicolumn{1}{c|}{\begin{tabular}[c]{@{}c@{}}1\\ (7.7\%)\end{tabular}} & \multicolumn{1}{c|}{0} & \multicolumn{1}{c|}{\begin{tabular}[c]{@{}c@{}}8\\ (61.5\%)\end{tabular}} & \multicolumn{1}{c|}{\begin{tabular}[c]{@{}c@{}}4\\ (30.8\%)\end{tabular}} \\ \hline
\end{tabular}%
}
\end{table}

\FloatBarrier
\subsection{Part 2: Evaluation of Recommendations}
The following figures display the results from questions that concerned the evaluation of recommendations received in Utsida. The first two questions asked participants to evaluate the recommendations they received, where the query was decided by the user's themselves.

\begin{figure}[H]
    \centering
    \begin{subfigure}[b]{0.4\textwidth}
        \begin{tikzpicture}[scale=0.6]
            \pie[sum = auto , after number =]{36/Yes, 4/No}
        \end{tikzpicture}    
        
        \caption{Question: Did the system recommend one or more \textbf{universities} which are relevant for you?}
        \label{fig:free_search_results_p1}
    \end{subfigure}
    ~ \qquad %add desired spacing between images, e. g. ~, \quad, \qquad, \hfill etc. 
      %(or a blank line to force the subfigure onto a new line)
    \begin{subfigure}[b]{0.4\textwidth}
        \begin{tikzpicture}[scale=0.6]
            \pie[sum = auto , after number =]{29/Yes, 11/No}
        \end{tikzpicture}  
       
        \caption{Question: Did the system recommend one or more \textbf{courses} which are relevant for you?}
        \label{fig:free_search_results_p2}
    \end{subfigure}
    \caption[Result from evaluation of recommendation]{Results from question asking to evaluate the recommendation on participants' self made query. (N = 40).}
    \label{fig:free_search_results}
\end{figure}

Two predefined queries, \textit{query 1} and \textit{query 2}, were evaluated by the participating students. The queries can be seen in Appendix \ref{appendix:pre_defined_search}. The degree of suitability ranges from \textit{1: Very poor} to \textit{5: Very good}.

\begin{figure}[h]
    \centering
    \begin{subfigure}[b]{0.4\textwidth}
        \begin{tikzpicture}[font=\small, trim axis left, trim axis right]
        \begin{axis}[
            height=5cm,
            width=7cm,
            ybar,
            bar width=20pt,
            xlabel={Degree of suitability (1-5)},
            ylabel={Number of answers},
            ymin=0,
            ytick={3,6,9,12,15,18,21,23},
            xtick=data,
            axis x line=bottom,
            axis y line=left,
            enlarge x limits=0.2,
            symbolic x coords={1, 2, 3, 4, 5},
            xticklabel style={anchor=base,yshift=-\baselineskip},
            nodes near coords={\pgfmathprintnumber\pgfplotspointmeta}
        ]
        
          \addplot[fill=bootstrapBlue] coordinates {
            (1, 0)
            (2, 1)
            (3, 9)
            (4, 22)
            (5, 8)
          };
        \end{axis}
    \end{tikzpicture}
        \caption{Results for \textbf{Query 1}}
        \label{fig:predesigned_1_p1}
    \end{subfigure}
    ~ \qquad %add desired spacing between images, e. g. ~, \quad, \qquad, \hfill etc. 
      %(or a blank line to force the subfigure onto a new line)
    \begin{subfigure}[b]{0.4\textwidth}
        \begin{tikzpicture}[font=\small, trim axis left, trim axis right]
        \begin{axis}[
            height=5cm,
            width=7cm,
            ybar,
            bar width=20pt,
            xlabel={Degree of suitability (1-5)},
            ylabel=\empty,
            ymin=0,
            ytick={3,6,9,12,15,18,21,23},
            xtick=data,
            axis x line=bottom,
            axis y line=left,
            enlarge x limits=0.2,
            symbolic x coords={1, 2, 3, 4, 5},
            xticklabel style={anchor=base,yshift=-\baselineskip},
            nodes near coords={\pgfmathprintnumber\pgfplotspointmeta}
        ]
        
          \addplot[fill=bootstrapBlue] coordinates {
            (1, 0)
            (2, 1)
            (3, 9)
            (4, 18)
            (5, 12)
          };
        \end{axis}
    \end{tikzpicture}
        \caption{Results for \textbf{Query 2}}
        \label{fig:predesigned_1_p2}
    \end{subfigure}
    \caption[Result for suitability of pre-defined recommendation]{Results from question: "How did the recommendation suit the original search parameters?" (N = 40)}
    \label{fig:predesigned_1}
\end{figure}

\begin{figure}[h]
    \centering
    \begin{subfigure}[b]{0.4\textwidth}
        \begin{tikzpicture}[scale=0.7]
            \pie[sum = auto , after number =]{24/Yes, 7/No, 9/Unsure}
        \end{tikzpicture}    
        
        \caption{Question: Did the system recommend relevant courses for one who studies \textbf{computer technology?}}
        \label{fig:predesigned_2_p1}
    \end{subfigure}
    ~ \qquad %add desired spacing between images, e. g. ~, \quad, \qquad, \hfill etc. 
      %(or a blank line to force the subfigure onto a new line)
    \begin{subfigure}[b]{0.4\textwidth}
        \begin{tikzpicture}[scale=0.7]
            \pie[sum = auto , after number =]{27/Yes, 2/No, 11/Unsure}
        \end{tikzpicture}  
       
        \caption{Question: Did the system recommend relevant courses for one who studies \textbf{electronics?}}
        \label{fig:predesigned_2_p2}
    \end{subfigure}
    \caption[Result for relevancy of pre-defined recommendation]{Results from question on relevancy of recommendation from \textbf{(a)} Query 1 and \textbf{(b)} Query 2 (N = 40)}
    \label{fig:predesigned_2}
\end{figure}


\FloatBarrier

\subsection{Open Question}

The open question in questionnaire 2 asked if the participant had any other comments on the site or process in general. 20 participants answered the question. A qualitative theme analysis \cite{oates2005researching} was performed and the findings concluded to be mostly positive with 11 out 20 having a positive theme. Five of the participant comments mentioned the site was non-intuitive, four thought it was easy to use, and three wrote Utsida could be time-saving. The full results of the theme analysis can be viewed in Appendix \ref{app:questionnaire2_open_question}.

\FloatBarrier
\section{Offline Experiment on the CBR-RS}

By using the score matrix given by Table \ref{tab:offline_test}, 20 different simulated user made queries (App. \ref{app:user_queries}) were evaluated. Each query was scored (App. \ref{app:full_offline_test_results}) on two different concept configurations, one being the similarity based retrieval used in the CBR-RS and the other being a simulated exact match search. The central tendency results are shown in Table \ref{tab:offline_test_results}.

\begin{table}[H]
\centering
\caption[Result from offline experiment]{Result from offline experiment, N=20}
\label{tab:offline_test_results}
\begin{tabulary}{\textwidth}{L|L|L|L|}
\cline{2-4}
                                                                           & Mean score & Std. Deviation & Std. Error Mean \\ \hline
\multicolumn{1}{|l|}{\cellcolor[HTML]{EFEFEF}Similarity based retrieval}   & 36.70 & 4.38           & .978            \\ \hline
\multicolumn{1}{|l|}{\cellcolor[HTML]{EFEFEF}Simulated exact match search} & 30.05 & 3.93           & .878            \\ \hline
\end{tabulary}
\end{table}

Table \ref{tab:offline_test_ttest} shows the results from the Paired T-test performed on the two result data sets from the offline experiment. The Sig. (2-tailed) value is the p-value. This value is lower than 0.05, indicating a significant difference between the two sample means. 

\begin{table}[H]
\centering
\caption{Result of the Paired T-test}
\label{tab:offline_test_ttest}
\begin{tabulary}{\textwidth}{ccc|c|c|ccc}
\cline{4-5}
 &  &  & \multicolumn{2}{c|}{\begin{tabular}[c]{@{}c@{}}95 \% Confidence Interval \\ of the Difference\end{tabular}} &  &  &  \\ \hline
\multicolumn{1}{|c|}{Mean} & \multicolumn{1}{c|}{\begin{tabular}[c]{@{}c@{}}Std. \\ Deviation\end{tabular}} & \begin{tabular}[c]{@{}c@{}}Std. \\ Error Mean\end{tabular} & Lower & Upper & \multicolumn{1}{c|}{t} & \multicolumn{1}{c|}{df} & \multicolumn{1}{c|}{\begin{tabular}[c]{@{}c@{}}Sig. (2-\\ tailed)\end{tabular}} \\ \hline
\multicolumn{1}{|c|}{6.65} & \multicolumn{1}{c|}{3.00} & .671 & 5.244 & 8.056 & \multicolumn{1}{c|}{9.897} & \multicolumn{1}{c|}{19} & \multicolumn{1}{c|}{0.000} \\ \hline
\end{tabulary}
\end{table}






\cleardoublepage