%===================================== CHAP 5 =================================

\chapter{Conclusion}
This chapter concludes the research performed and gives insight into possible further work that can be done on the subject. 

\section{Conclusion}
This study have found that a system for recommending courses and universities could positively influence the motivation of students for going on an exchange program. While it is not possible to fully conclude that the results are given for the whole population they highly indicate a positive effect with 85.8\% average agreement on statements regarding motivation. CBR has also been tested on this domain and was found to be suitable for giving recommendation on universities and courses. While the recommendations alone can not fully be used by the students as a definite solution they provide an inspiration and can be the step that starts the process. If applied in a real setting Utsida would also likely reduce the time used for both students and advisers when approving courses for exchange programs. This would mean more time for advisers to help students in actual need of help and use less time on students that just need to go through the administrational process. This study therefore highly recommends NTNU and other educational institutes to further invest in projects that could digitalise the process of applying for an exchange program.



\section{Further work}
All though the system was completed to a degree needed to complete this study, more can be done both in terms of continuing the development of the system, and conducting research with it. The next sections include features which would be meaningful to implement in the system, improvements which could be made to the CBRS, and ideas for doing further research with Utsida.

\subsection{System Features}
Suggestions for improving the system arose throughout the development process. Interviews with exchange course advisers, feedback from usability testing on both students and exchange course advisers, and textual answers from the questionnaire all proposed interesting and meaningful features which could be included in the system. Some of these features were implemented, but many were left out because they were out of the scope for this research. The following section describes the most noteworthy suggestions which would be meaningful to implement in the system.

\begin{description}
    \item[Let advisers make comments on course match applications:] Currently, a course match in Utsida is either approved or disapproved. These matches often included premises to be approved, or they may only be valid for a certain group of students. By implementing a comment functionality for advisers, they could denote this information, making it more useful for students reading it.
    \item[Automatically expire course matches which are too old:] Course matches tend to be irrelevant if they are too old, due to changes at universities. There could be an automatic review on the course matches which could for example expire old ones.
    \item[Implement filters on the course matches table:] The course matches table currently includes a search bar, but no filtering. Some students wanted to filter these matches further for example on institute or faculty. This would make it easier to find relevant course matches on the site.
\end{description}

\subsection{Expanding and Improving the Model of the CBRS}
- More factors (Cost of living, weather/climate)
- Adapt to new experience reports with bundled course matches
- Gain feedback on recommendation relevance
- search with courses

\subsection{Research}
Due to lack of time and resources such as money and participants, there was no extensive research done to test Utsida in a real world situation. The following suggestions include ways to see Utsida's effect in real world cases.

\subsubsection{Testing Utsida in an application Process}
By continuing working on Utsida, a working prototype be used in the application process for a collection of students. Data could then be collected to find out what these students thought of the application process, in comparison to the students who went through the standard application process. This data could be used to argue the inclusion of the system in The Office of International Relations' process, and further document its effect.


\subsubsection{Observing Real Effects of Utsida}
If Utsida was to be included in application process for an exchange program by The Office of International Relations at NTNU, it would have been interesting to study the number of students applying with the system in use compared to current numbers. If there had been a significant increase of students applying after Utsida was included, it would suggest that the system helped NTNU with its goal of sending more of its students on exchange programs.



\cleardoublepage