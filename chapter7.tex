\chapter{Discussion}

This chapter interprets the results presented in the previous chapter and discusses them with regards to the research questions and goals. Identified limitations of the project is also presented.

\section{Motivational Effect}

Goal 1 was to \enquote{Create a prototype that improves the motivation for students at NTNU to apply for an exchange program}. This study created that prototype and named it Utsida. Utsida's motivational effect was measured by conducting a user study with an accompanying questionnaire. The questionnaire shows a strong positive influence on motivation, with 23 out of 27 participating students that had not been on exchange agreeing (i.e. answers of either 4: Agree or 5: Strongly agree) that their motivation would likely increase if Utsida was in use. The frequencies also shows that the answers did not vary from the mode of 4, indicating that most of the participants had similar views. More than half of the participants (18 out of 27) further reviewed and agreed that Utsida was easy to use. Utsida should ideally have a higher agreement on the usability, but it is considered acceptable because of early prototype stage of Utsida. Furthermore, a Cronbach's alpha value of 0.916 on the Likert scale of participating students that have not been exchange, and 0.855 for those that had, shows a strong sign of internal consistency in questionnaire 2, with values over 0.7 considered satisfactory \cite{bland1997statistics}.

The mode for all questions in the first part of questionnaire 2 resides between \textit{4: Agree} and \textit{5: Strongly Agree}. These statistics implies that the participants had a largely positive experience with Utsida. 37 out of the 40 participating students agreed that Utsida could contribute to an increased numbers of students that participate in an exchange program. This correlates with the results of a study done by NTNU's Office of International Relations (OIR) \cite{intersek_report} where exchange students were asked what factors could contribute to an increased number of exchange students. Utsida specifically targets the top factors in the study with 85\% of the participants selecting \enquote{lists of previously approved courses} and 80\% selecting \enquote{lists over recommended universities with approved courses}. 39 out of 40 the participants of questionnaire 2 would also recommend OIR to include Utsida in their application process. This indicates a high user confidence in the system, however, this question could have been especially vulnerable to bias, as further discussed in the limitations (sec. \ref{sec:limitations}).

The results from questionnaire 1 (Tab. \ref{tab:attribute_ranking}) concluded that the most important factors of motivation is the study language, social quality and academic quality. These results are supported by the the results of the \emph{push-pull} factors of Mazzarol and Soutar \cite{mazzarol2002push} which identifies academic quality, social links and language as important influences. The six most important factors identified in questionnaire 1 were included in the CBR-RS to ensure high relevance of attributes. The open question also identified that general information and an easier application process are the important areas to focus on. This further increase the confidence in that Utsida, which targets these areas, would have a positive effect on students' motivation for applying for an exchange program.



\section{Suitability of CBR Methodology}

Goal 2 was to \enquote{Use the CBR methodology to give relevant recommendations on exchange program universities and courses}. The goal was accomplished by creating a CBR-RS that uses myCBR and by: converting exchange students' experience reports to cases, setting appropriate similarity measures based on literature and user testing, and giving the attributes weights based on the results of questionnaire 1. The offline experiment concluded that using similarity-based recommendation give more relevant results than traditional exact matching search. The conducted paired T-test on the results had a p-value less than 0.05, indicating that the results are statistically significant and that the difference is highly unlikely to be created by chance. The closest thing NTNU currently have to Utsida is the OIR's site\footnote{Experience report database: \url{https://www.ntnu.no/studier/studier_i_utlandet/rapport/}} for publishing and finding students' experience reports from exchange programs. This site uses traditional exact matching techniques to search through the reports. Thus, based on the results of the offline experiment, it should be easier to find relevant experiences in Utsida than in OIR's search system. 

The results of the evaluation of recommendations part of questionnaire 2 are highly positive. The results shows that the majority of the participants received relevant recommendations with their queries. However, a clear tendency is that more participating students received relevant recommendations on universities (36 out of 40) than courses (29 out of 40). This tendency might be due to universities being a far more general factor than courses. A university can be of interest for many, while a course is more strictly connected to a particular field of study and year. Furthermore, the courses in a recommendation are considered novel recommendations, which are recommendations for items the user did not know about \cite{shani2011evaluating}. Therefore, it could be difficult for the participant to evaluate the relevancy of unknown courses. Since the experience reports also are mostly written in free text, some course titles may in unknown foreign languages. In addition, the courses do not have any other description than the title, making it harder for students to know if the courses are relevant to them or not. 

The results of the questions asking the participants to evaluate the recommendations for the two predefined show highly positive tendencies. The relevancy of the recommendations, as a whole, were deemed suitable (i.e. answered either 4: Good or 5: Very good) by 30 out of 40 participants for both queries. This shows that the CBR-RS, in most cases, gives relevant recommendations for a input query. Furthermore, among the 40 participants, 24 thought Utsida recommended relevant courses for the first predefined query, and 27 for the second predefined query. These results suggests that the CBR-RS would likely recommend relevant courses to students similar to the predefined queries. An average of 10 out 40 participants were unsure to deem the recommended courses relevant for each query. This uncertainty could come from many participants not having enough knowledge on the field of study, and are, therefore, not able to evaluate the relevancy of the courses. In hindsight, the questions could have been constructed without the \enquote{Unsure} option, which do not contain much value for this research project. However, removing the option might have introduced more bias in the \enquote{Yes/No} answers as the participant might be unsure what to select. 

\section{Limitations}\label{sec:limitations}

Some limitations with the conducted research was discovered during the project duration. The most significant limitation was the small sample size for the questionnaires. Even though both questionnaires got a reasonable amount of response, questionnaire 1 with 84 answers and questionnaire 2 with 40 answers, they were both below the goal and optimal sample size. A confidence level of 95\% resulted in a high margin of error of 10.6\% on questionnaire 1 and a even higher margin of error of 15.5\% on questionnaire 2. However, the result data will show a positive tendency even in the worst case of the population answers being on the low end of the margin. 

Utsida had 165 unique registered users during the online user test, but only 40 of them (24\%) completed the full test and answered questionnaire 2. This suggest that the user study may have been too complex and time-consuming and therefore made many participants withdraw before finishing. Furthermore, the questionnaires should have undergone more rigorous pilot testing with experts on the field of questionnaire design to further reduce the risk of bias influence. The question on whether the participant would recommend OIR to include Utsida in their exchange application process was found to be especially vulnerable to acquiescence bias and should have been designed differently.

\section{Research Questions}

\subsubsection{RQ1: \enquote{\textit{What effect would an information system for recommending and assisting exchange course selection have on students' motivation for doing a study abroad- or student exchange program at NTNU?}}}

The results from evaluating Utsida, which is an information system that recommends and assists the exchange course selections for students, show a highly positive effect on students' motivations for doing an exchange program. In addition, the results also indicate that Utsida may, if used, lead to more students applying for exchange programs. The majority of the answers in questionnaire 2 were positive, with an average agreement on statements regarding motivation of 85.8\%. Most noteworthy, 37 out of 40 participating students agreed that Utsida could increase the number of students that apply for an exchange program and 23 out of 27 participants that had not been on exchange agreed that their motivation for exchange would increase with Utsida in use.

\subsubsection{RQ2: \enquote{\textit{How suitable is the Case-Based Reasoning methodology for recommending relevant universities and subjects for a study abroad- or study exchange program?}}}

The positive responses in the user study, and the significant results of the offline experiment, indicates that the CBR methodology is suitable for giving recommendations on courses and universities. The majority of participating students in the user study received relevant recommendations in Utsida. In detail, 36 out of 40 participants reviewed the university recommendations to be relevant, while 29 out of 40 participants received relevant recommendations on courses. However, the recommendations tended to be used more as a tool for course inspiration than to give the definitive solutions. This is also more in accordance with giving recommendations; a recommendation may not have a suitable solution, but the user is left with new inspiration and ideas. 