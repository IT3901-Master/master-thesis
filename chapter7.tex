\chapter{Discussion and Conclusion}

The evaluations of the prototype yielded the results needed to answer both research questions, and whether the goals of the project were met. This chapter presents a discussion on how the results answer the research questions, presents limitations found, recommendations by the authors, and a conclusion of the thesis.

\section{RQ1: Motivational effect}

\enquote{\textit{What effect would a recommender system for assisting exchange course selection have on students' motivation for doing a study abroad- or study exchange program at NTNU?}}

Goal 1 was to \enquote{Create an information system that improves the motivation for students at NTNU to go on an exchange program}. This study created that information system and named it Utsida. To measure its motivational effect, and answer RQ1, a group of students were offered to test Utsida and answer a coherent questionnaire. The questionnaire showed a strong positive influence on motivation. With 23 out of 27 students that had not been on exchange agreeing that their motivation would most likely increase if Utsida was in use. The frequencies also shows that the answers did not spread much from the mode of 4, indicating that most of the participants had similar views. The system was also reviewed as easy to use by most of the participants, with 66\% agreement. All though it was desired to have a higher percentage of users who perceived the system as easy to use, it is considered acceptable, because it is difficult to determine the ease of use for a system in the early stages of development. A Cronbach's alpha value of 0.916 on the Likert scale of students that have not been exchange and 0.855 for those that had shows a strong sign of internal consistency in questionnaire 2, with values over 0.7 considered satisfactory \cite{bland1997statistics}.

The mode for all questions in the first part of questionnaire 2 resides between \textit{4: Agree} and \textit{5: Strongly Agree}. These statistics implies that the students had a largely positive experience of Utsida. Over 90\% of both of the groups agree to the statement that Utsida could contribute to an increased numbers of students who choose to participate in an exchange program. This correlates with the results of the study \cite{intersek_report} done by NTNU's Office of International Relations where students were asked what factors could contribute to an increased number of exchange students. Utsida specifically targets the top factors in the study with lists of previously approved courses (85\%) and lists over recommended universities with approved courses (80\%).

The results from questionnaire 1 concluded that the most important factors of motivation is the study language, social quality and academic quality (see Table \ref{tab:attribute_ranking}). These results are supported by the the results of the \emph{push-pull} factors of Mazzarol and Soutar \cite{mazzarol2002push} which identifies academic quality, social links and language as important influences. The six most important factors identified in the study were included in the recommender system to ensure high relevance of attributes. The open question also identified that information on courses, general information and an easier application process are important areas to focus on. This further increase the confidence in that a system like Utsida, which targets these areas, would have a positive effect on students' motivation for applying for an exchange program.

\section{RQ2: Suitability of CBR Methodology}

\enquote{\textit{How suitable is the Case-Based Reasoning methodology for recommending relevant universities and subjects for a study abroad- or study exchange program?}}

Goal 2 was to \enquote{Use the CBR methodology to give relevant recommendations on exchange program universities and courses for students at NTNU}. The goal was accomplished by creating a CBR-RS with the use of myCBR, converting exchange students' experience reports to cases for the case-base, setting appropriate similarity measures based on literature and adjusting the attributes weights based on the results of questionnaire 1. A combination of an offline experiment and an online user test with an accompanying questionnaire was conducted to answer RQ2. 

The offline experiment concluded that using similarity-based retrieval yields more relevant results than traditional exact matching techniques, while also producing feasible results based on the score matrix which was designed. The same score matrix and evaluation method were used on both types of searching, and the conducted paired T-test had a p-value less than 0.05, meaning that the results are statistically significant and that the difference is highly unlikely to be created by chance. The closest thing NTNU currently have to Utsida is the OIR's site for publishing and finding students' experience reports from exchange programs. This site uses traditional exact matching techniques to search through the reports. Thus, based on the results of the offline experiment, it should be easier to find relevant experiences in Utsida than in this system. 

In the online user test, evaluated by questionnaire 2, the results are considered highly positive. With the results showing that the majority (36 out of 40) of the participants were able to obtain relevant recommendations of some kind with the query the students composed themselves. However, a clear tendency is that more students got recommended relevant universities than courses. The tendency is likely due to universities being a far more general factor than courses. A university can be of interest for many, while a course is more strictly connected to a particular study field and year. Furthermore, because there are little constraints on the text in the cases, many of them included courses written in foreign languages, and many lack a proper description of them, making it harder for students to know if the courses are relevant to them or not. For the questions asking the participants to evaluate the recommendations for two predesigned queries (Figure \ref{fig:predesigned_1} and Figure \ref{fig:predesigned_2}), the results also show highly positive tendencies, yet somewhat more spread. Figure \ref{fig:predesigned_2} shows that roughly a quarter of the respondents were unsure to deem the recommended courses relevant for each query. To evaluate the recommendations for a predesigned query, the evaluator has to have some domain knowledge on specifically the student department in the query and what kind of courses which are relevant for a student of that department. The \enquote{Unsure} option was added to the questionnaire because of the uncertainty. In hindsight, these questions could have been constructed in a better way, to avoid many replies yielding \enquote{Unsure}, and therefore not be given much value. 

Based on the positive response on the online user test, and results of the offline experiment, we can conclude with certainty that the CBR methodology is suitable, and indicated by the sample size of questionnaire 2, yields relevant and accurate recommendations for the majority of students. However, at least in the case of this study, it functions at most as a tool for inspiration. With the amount of data used in the case base, it is highly unlikely for a student to find a replacement for all courses they would take at NTNU at the same university. But, this is also more in line with the word \enquote{recommendation}, show relevant recommendations which help the student in the right direction by inspiring them with the successful experiences of other students. 


\section{Limitations}
Some limitations with this research was discovered during the process. The most important being the sample size of the population for the questionnaires. While both questionnaires got a decent amount of response; questionnaire 1 with 84 answers and questionnaire 2 with 40 answers. It was below both the goal and optimal sample size. With a confidence level of 95\% this resulted in a margin of error (MoE) of 10.6\% on questionnaire 1 and a MoE of 15.5\% on questionnaire 2. The result data does however still have a highly positive tendency with average responses to all Likert type questions residing between \textit{4: Agree} and \textit{5: Strongly Agree}.

Utsida had 165 unique registered users during the online user test, only 40 of them (24\%) completed the full test with questionnaire 2. This may indicate that the whole structure of the test and the questionnaire was too complex, which lead to many students gave up before finishing it.  Furthermore, to avoid more bias, the questionnaires should have undergone more rigorous pilot testing with experts on the field of questionnaire design, this was not possible due to time limits and cost. The question on whether the students would recommend OIR to include Utsida in their exchange application process was found to be especially vulnerable to acquiescence bias and should have been redesigned. This question had 39 replies answering with \enquote{Yes} out of the total 40 replies.

\section{Conclusion}

This research project has answered RQ1 and found that a system for recommending courses and universities would positively influence the motivation of students for going on an exchange program. The measured motivational effect of Utsida is very positive with an average agreement on statements regarding the motivation of 85.8\% with a confidence level of 95\% and margin of error 15.5\%. While this margin of error is relatively high due to a small sample size, the results of this project still highly indicate a positive effect.

Furthermore, RQ2 has been anwered and CBR is found to be highly suitable for giving recommendations on universities and courses. While the recommendations alone can not fully be used by the students as an independent solution, they might give the students enough inspiration and motivation to initiate the process. If applied in a real setting, Utsida would also likely reduce the time used for both students and advisers when approving courses for exchange programs. This would mean more time for advisers to help students in actual need of help and use less time on students that just need to go through the administration process.

Based on the results that show a very positive attitude for Utsida, it is highly recommended that NTNU invests further in digitalizing the whole- or part of the application process for exchange programs. An investment could help NTNU reach their goal of 40\% of students going on exchange programs and contribute to the overall goal in Norway of 20\% student mobility. This project has also shown that exchange reports submitted by students can be used in other useful ways. Therefore the data from the experience reports should be public to use (e.g. as an API), and be formatted appropriately so that students and others can utilize them in ways that may give a benefit to NTNU. The results from this project are not only applicable to NTNU alone, but could also be used as an inspiration for similar projects at other educational institutions. 