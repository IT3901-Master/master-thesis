\chapter{Preliminary Research} \label{chap:2}

Before producing the research questions and deciding on the specific research methods to be used, it was important to understand the problem domain and related work done in the field. This chapter details that process and the knowledge gained through interviews, meetings and literature review. The chapter is concluded with the identified system requirements and research gap.

\section{Interviews}

Interviews and informal meetings were conducted to understand the problem domain and gain information on the current approach of applying for an exchange program from NTNU. Two advisers from the Computer Science Department at NTNU (IDI) were interviewed, and a meeting was conducted with the Office of International Relations at NTNU (OIR). These engagements were also conducted to gain valuable feedback on possible difficulties and limitations that might be encountered in the project.

\subsection{Student Advisers} 

Student advisers were interviewed to gain a better understanding of their role in the process of approving courses for an exchange program. In the context of this process, student advisers mostly work on reviewing courses found by students and gives recommendations on courses to take, based on the students' study plan. Advisers also give information to students on partner universities and on the process as a whole. Several problem areas were noted by the advisers. Firstly, a high workload in peak application periods makes it difficult to spend enough time reviewing students' courses. Secondly, a generally large number of students makes it difficult for a personal follow-up throughout the exchange program duration. Finally, the current process of approving exchange courses is arduous. The advisers were highly positive to a research project which could find ways to resolve some of these issues. Features suggested by advisers included digital course applications, digital lists of approved courses for each department, and automatic expiration of course approvals in this list based on approval date.

\subsection{The Office of International Relations}

The main reason for meeting with the staff of the OIR was to gain insight into the current process of applying for exchange programs at NTNU. Another reason was to gain access to the exchange experience report database that OIR maintains. These reports would be highly valuable for giving the recommendations in the prototype. The database access was, however, found to be restricted because it stores sensitive personal information. OIR instead gave permission for this project to extract the reports directly from the website. The OIR also presented, during the meeting, the results of a study on student motivation they conducted in the spring of 2016 that were useful for determining focus areas of the project. Furthermore, they noted some limitations the project might encounter, such as different practices at various departments and little information on unpopular exchange locations. The OIR suggested many features, among these were support for differences in study points of courses, and recommendation of courses and universities at locations with fewer students.

\subsection{Previous Exchange Students}

Because one of the authors has previously been on exchange, some of the difficulties of the current process were already identified. It was, however, useful to interview students from different departments and with other experiences to gain a better understanding of the problem domain. The focus of the interviews was to ask questions on issues with the current process and to brainstorm possible improvements. Significant issues noted by the students were: concern of courses not being approved after an exchange period, difficulty changing courses during the exchange period, and the tedious manual process before an exchange period.

The proposed features from students were: A map of where other students have done an exchange program, combining courses from NTNU with possible replacements at a foreign university and recommendation of universities and courses based on the preferences of each student. 

\section{Current Approach}

%Student exchange programs from NTNU only exist at educational institutions that have a formal student exchange agreement with NTNU.

The following description is based on the current approach at the Computer Science Department (IDI) at NTNU. It may, therefore, be some differences at other departments. See Figure \ref{ntnu_process} for a short overview of the whole application process. The boxes colored in green are the targeted problem areas of this research project.

\begin{figure}[H]
  \centering

\begin{tikzpicture}[
  level 1/.style={sibling distance=30mm},
  edge from parent/.style={->,draw},
  >=latex]

% root of the the initial tree, level 1
\node[root] {Apply for exchange}
% The first level, as children of the initial tree
  child {node[level 2] (c1) {1. Select location}}
  child {node[level 2] (c2) {2. Make application}}
  child {node[level 2] (c3) {3. During the stay}}
  child {node[level 2] (c4) {4. After the stay}};

% The second level, relatively positioned nodes
\begin{scope}[every node/.style={level 3}]
\node [below=4pt of c1, xshift=5pt] (c11) {\small 1.1 Talk to adviser};
\node [below=4pt of c11, color=prettyGreen,text=black] (c12) {\small 1.2 Find experiences};
\node [below=4pt of c12, color=prettyGreen,text=black] (c13) {\small 1.3 Decide on location};
\node [below=4pt of c2, xshift=5pt,color=prettyGreen,text=black] (c21) {\small 2.1 Pre-approve courses};
\node [below=4pt of c21] (c22) {\small 2.2 Create exchange application};
\node [below=4pt of c22] (c23) {\small 2.3 Upload and send application};

\node [below=4pt of c3, xshift=5pt] (c31) {\small 3.1 Write experience report};
\node [below=4pt of c31,color=prettyGreen,text=black] (c32) {\small 3.2 Contact adviser in case of course change};

\node [below=4pt of c4, xshift=5pt,color=prettyGreen,text=black] (c41) {\small 4.1 Create final course approval form};
\node [below=4pt of c41,color=prettyGreen,text=black] (c42) {\small 4.2 Send in final approval form};

\end{scope}

\end{tikzpicture}

\caption[Current approach to apply for an exchange program]{Current approach to apply for an exchange program at NTNU\protect\footnotemark}
\label{ntnu_process}
\end{figure}

\footnotetext{https://innsida.ntnu.no/utenlandsstudier}

Students are first required to find and choose the university where they want to do an exchange program. There is a considerable number of possible choices, and NTNU only gives information on universities with an cooperation agreement. After the university is chosen, the students have to find courses that can be approved as a replacement for the courses originally planned at NTNU. Most of the departments do not have a system where students can enter their replacements or find the previously approved ones. One exception is IDI, which utilize a digital table to make it easier for students to find course replacements. However, manually maintaining the table is tedious work and it lacks useful features such data validation. 

Throughout the course of an exchange program, the students often have to change their courses due to unforeseen circumstances, making the original approval invalid. To gain a new approval, a student usually has to send an email to the adviser of their department to get the new courses approved. If this process takes a while, and the course registration due date is close, the student may have to take a course without approval from NTNU, risking not getting it approved at the end of the exchange program.

\section{Related Work} \label{related_work}

Mazzarol and Soutar \cite{mazzarol2002push} found that students' general motivation is influenced by the amount of available information on a university and its courses. Among the several factors which were reviewed, the \enquote{knowledge and awareness} factor proved to be the most influencing one for choosing an international study location. Therefore, when researching how an IS can improve the motivation of students to participate in an exchange program, focusing on providing knowledge and awareness is important. 

Furthermore, the OIR conducted a study on 464 exchange students in 2016 to investigate what factors could help increase the motivation to apply for an exchange program \cite{intersek_report}. Among the 464 participating students, 84\% answered that a list of previously approved exchange courses could increase the number of students that go on exchange, and 80\% answered that pre-approved course packages could increase the numbers. These results indicates that an IS which target these areas would be important to increase the number of students that apply for an exchange program.

Several approaches have been made to replace the manual process of choosing courses with an IS. One example is by using a decision support system (DSS) that advises students on their course selection based on their program requirements and the course's prerequisites, as done at the University of Dhaka \cite{roushan2014university}. Furthermore, student course recommendation can also be done intelligently by using recommendation engines and data mining techniques to give relevant results. Sherpa \cite{bramucci2012sherpa} is a system that has been made especially for the goal of providing course recommendations. Data mining techniques can also be used to predict whether a student will fail or pass on a course \cite{vialardi2009recommendation}.

Recommender systems are used extensively in both research and commercial products. These systems provide several possible solutions that might be viable for the user to choose. Several published research papers have explored different approaches to recommender systems \cite{mulyana2015case}\cite{quijano2011happy}. When designing a recommender system it is important to know which data that should be used as attributes. Cubillo et al. \cite{maria2006international} identified the different decision making factors for international students and their grouping. The five main groups were city effect, country image effect, personal reasons, institution image and program evaluation. One way to implement a recommender system is by utilizing case-based reasoning (CBR).

CBR is today both a recognized and well-established method in several fields of sciences, such as health science \cite{begum2011case}. Because CBR is proven to be a suitable methodology for complicated problems where uncertainty is involved \cite{richter2013case}, it has been applied to problems such as diagnosing chronic diseases in combination with data mining \cite{huang2007integrating}, and finding the most suitable study program at a senior high school \cite{mulyana2015case}. 

Case-based reasoning recommender Systems (CBR-RS) are also commonly used in decision problems, typically when there is not one exact answer, but rather several different good answers. Many types of problems can be solved using CBR-RS systems. Examples of such systems are: MOOC-Rec \cite{bousbahi2015mooc} a system which finds the most suitable Massive Open Online Courses for e-learning, DieToRecs \cite{fesenmaier2003dietorecs} which recommends travel destinations and activities, and helps planning trips, and Entree \cite{trewin2000knowledge}, which recommends restaurants based on a users preferences such as price, type of food and atmosphere. 
\section{Conclusion}

This section presents the system requirements for the prototype that were identified in the preliminary research as possible motivational factors, and the identified research gap.

\subsection{System Requirements}\label{sec:requirements}

The different needs and possible contributions identified in the preliminary research through literature review, interviews and analyzing the current approach are shown in the requirements list, Table \ref{tab:feature_list}. In the survey \cite{intersek_report} sent out by the OIR to students that have been on exchange programs, 85\% of 464 students stated that a list containing earlier approved course matches would increase the number of students going on study exchange programs. This functionality, therefore, felt especially important to include as a requirement. The findings were also used to help define the research goals and questions (sec. \ref{RQ}). 

\begin{table}[h]
\centering
\caption[System requirements]{System requirements derived from preliminary research}
\label{tab:feature_list}
\begin{tabulary}{\textwidth}{|L|L|L|L|}
\hline
\textbf{\#} & \textbf{Requirement}                                                                                                & \textbf{Priority} & \textbf{Source}                                                                     \\ \hline \hline
1  & Recommend universities and courses                                                               & 1        & Students, OIR                                                                    \\ \hline
2  & List previously approved course matches                                                             & 2        & Students, advisers                                                          \\ \hline
3  & \begin{tabular}[c]{@{}l@{}} Support matching of exchange courses \\ with NTNU courses \end{tabular}                                                             & 2        & Students                                                                    \\ \hline
4  & Handle course approval applications                                                              & 3        & Advisers                                                                    \\ \hline
5   & \begin{tabular}[c]{@{}l@{}}Show number of exchange students that have \\ been in each country with a world map\end{tabular} & 4        & Students                                                                    \\ \hline
\end{tabulary}
\end{table}

\subsection{The Identified Research Gap}

Among the many applications utilizing the CBR methodology, none has covered the issue of finding the most suitable country, university and courses when going on an exchange program. This gap in knowledge combined with the needs identified in the system requirements is what this project will target through the research questions and goals (sec. \ref{RQ}).  


