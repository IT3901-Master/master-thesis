\chapter{Preliminary Research} \label{chap:2}
Before developing the research questions and deciding on specific research methods it was important to understand the problem context and related work done in the field. This chapter details that process and the knowledge gained.  

\section{Interviews}
Interviews and unofficial meetings were conducted to understand the problem domain and gain information on the current process for applying for an exchange program at NTNU. Two advisers from the Computer Science Department (IDI) were interviewed, and a meeting was conducted with the Office of International Relations. The goal of these meetings was to introduce the groups to our research goals. In addition it was also used as a method to gain valuable feedback on possible difficulties and limitations that might be encountered during implementation and testing. 

\subsection{Student Advisers} Student advisers were interviewed to gain a better understanding on their role in the process of approving courses. In the context exchange programs, study advisers mostly work on reviewing the courses found by students and giving them possible recommendations on courses to take so it fits the students study plan. They also give information to students on universities with partnerships and the process as a whole. Some of the difficulties noted by the advisers was a high workload in peak periods for applications, high number of students makes it difficult for a personal follow-up and difficulties comparing the courses proposed by students. They were highly positive to a possible study that could find ways to resolve some of these issues. Features proposed by advisers were online course applications and follow-up of these applications, list of approved courses at a university in a specific department and automatic expiration of course approval based on approval date. 

\subsection{Office of International Relations}
The reason for meeting with staff at the Office of International Relations at NTNU was to gain insight into their large knowledge on the particular field. Another goal was find out if advisers from different faculties worked together and cooperated. During the meeting the results of the study on student motivation the Office of International Relations conducted in spring 2016 was presented. Furthermore, we discussed possible limitations with the proposed project. Such as low amount of information on exchange locations where few students have been before, and the many different rules and customs at different departments. The Office of International Relations suggested features such as dynamic study point allocation, recommendation of courses and universities at locations with fewer students.

The Office on International Relations maintains the database of experiences reported by students that have been on exchange. This data is highly valuable for the recommendation engine and gaining access to the database was part of the meeting points. The database access was however found to be limited as it included personal information, therefore the office of international relations instead granted permission for extracting the data directly from the HTML sites. 

\subsection{Exchange students}
Since we had one author who had been on exchange previously, some of the difficulties of the current process was already investigated. It was however useful to include students from other departments and with different experiences to gain better knowledge on the problem domain. The interviews with students were focused on their issues with the current process and brainstorming ideas on how the process could be improved. Major findings included insecurity that courses would get the final approval when arriving back home after an exchange period, difficulty changing courses during the exchange period, and the straining manual process before an exchange period.

The proposed features from students included a map of where other students have done an exchange program, combining courses from NTNU with their respective match at the foreign university and recommendation of universities and courses specific to the preference of each student. 


\section{Current approach}
Our description is based on the current approach at the Computer Science Department (IDI) at NTNU, it may therefore be some differences between the different departments. The current process for choosing and approving exchange courses is mostly done manually at NTNU. See Figure \ref{ntnu_process} for a short overview of the whole application process. The boxes in green are the targeted problem areas of this research. The course approval process is required both before and after participating in an exchange program.

\begin{figure}[H]
  \centering

\begin{tikzpicture}[
  level 1/.style={sibling distance=30mm},
  edge from parent/.style={->,draw},
  >=latex]

% root of the the initial tree, level 1
\node[root] {Applying for exchange}
% The first level, as children of the initial tree
  child {node[level 2] (c1) {1. Select location}}
  child {node[level 2] (c2) {2. Make application}}
  child {node[level 2] (c3) {3. During the stay}}
  child {node[level 2] (c4) {4. After the stay}};

% The second level, relatively positioned nodes
\begin{scope}[every node/.style={level 3}]
\node [below=4pt of c1, xshift=5pt] (c11) {1.1 Talk to advisor};
\node [below=4pt of c11, color=prettyGreen,text=black] (c12) {1.2 Find experiences};
\node [below=4pt of c12, color=prettyGreen,text=black] (c13) {1.3 Decide on location};
\node [below=4pt of c2, xshift=5pt,color=prettyGreen,text=black] (c21) {2.1 Pre-approve courses};
\node [below=4pt of c21] (c22) {2.2 Create exchange application};
\node [below=4pt of c22] (c23) {2.3 Upload and send application};

\node [below=4pt of c3, xshift=5pt] (c31) {3.1 Write experience report};
\node [below=4pt of c31,color=prettyGreen,text=black] (c32) {3.2 Contact advisor in case of course change};

\node [below=4pt of c4, xshift=5pt,color=prettyGreen,text=black] (c41) {4.1 Create final course approval form};
\node [below=4pt of c41,color=prettyGreen,text=black] (c42) {4.2 Send in final approval form};

\end{scope}

\end{tikzpicture}

\caption{The current exchange program application process at NTNU \cite{ntnu_exchange}\label{ntnu_process}}
\end{figure}

Students are first required to find the university were they want to do an exchange program. The possible choices are nearly limitless  with information only on universities that have institutional cooperation with NTNU. After the university is chosen, the students has to find courses at the foreign university that can be approved as a replacement for the courses originally planned at NTNU. Most of the departments don't have a system where students can enter their replacements or find the previously approved ones. One of the exceptions however is the Department of Information and Computer Science which utilize a digital table for entering courses. While this table in many ways make it easier for the students to select their university and courses, the data entry does not include useful features such data validations. Maintaining a manual table is also tedious work. 

Throughout the course of an exchange program the students often has to change their courses due to unforeseen circumstances, making the original approval invalid. To gain a new approval a student usually have to send an email to the adviser of their department to get the new courses approved. If this process takes a while, and the course selection due date is close, the student may have to decide to register in a course without approval from NTNU, risking not getting it approved at the exchange programs end. 


\section{Related work} \label{related_work}

Mazzarol and Soutar \cite{mazzarol2002push} found that students' general motivation is influenced by the amount of information on an university and its courses. Among the several factors which was reviewed, the \enquote{knowledge and awareness} factor proved to be the most influencing one for choosing an international study location. Therefore, when researching how an IS can improve the motivation of students to participate in an exchange program, improving knowledge and awareness is key. 

The NTNU Office of International Relationships conducted a study on 464 of their exchange students in 2016 to investigate what factors could help increase the motivation for participating in an exchange program \cite{intersek_report}. Among the 464 participating students 84\% answered that a list of previously approved exchange courses could increase the number of students that go on exchange and 80\% answered that pre-approved course packages could increase the numbers. This shows the importance and possible usefulness of an IS that targets these areas. 

Several approaches has been made to replace the manual process of choosing courses with an IS. One example is by using a decision support system (DSS) that advises the students on their course selection based on their program requirements and the course's prerequisites, as done at the University of Dhaka \cite{roushan2014university}. Furthermore, student course recommendation can be done intelligently by using recommendation engines and data mining techniques to give relevant results. Sherpa \cite{bramucci2012sherpa} is a system that has been made especially for the goal of giving course recommendations. Data mining techniques can also be used to predict whether a student will fail or pass on a course \cite{vialardi2009recommendation}.

CBR is today both a recognized and well-established method in several fields of sciences; such as health science \cite{begum2011case}. Because CBR is proven to be a suitable methodology for complicated problems where uncertainty is involved \cite{richter2013case}, it has been applied to problems such as diagnosing chronically diseases in combination with data mining \cite{huang2007integrating}, and finding the most suitable study program at a senior high school\cite{mulyana2015case}. Case-Based Recommender Systems, which are introduced in section \ref{sec:case_based_recommender_systems} are also commonly used in decision problems, typically when there's not one true answer, but rather several different good answers. For instance was such a system was developed to find the most suitable Massive Open Online Courses (MOOCs) for e-learning.\cite{bousbahi2015mooc}. 

Designing a recommender system requires knowledge on the relevant data to use as attributes. A study was performed by Cubillo et al. (2006)\cite{maria2006international} that identifies the different important decision making factors for international students and their grouping. The five main groups identified were city effect, country image effect, personal reasons, institution image and programme evaluation.    


\section{Conclusion}
Based on the conducted interviews and acquired knowledge on NTNU's current process for applying for an exchange program, a set of features were chosen to be included in the system to be developed. By reviewing related work, the knowledge gap this research will target has also been identified.

\subsection{System Requirements}\label{sec:requirements}
The possible contributions and needs identified in the preliminary research through literature review, interviews and knowledge on the current approach are shown in the requirements Table \ref{tab:feature_list}. The findings was also used to help define the research goals and questions, see section \ref{RQ}. 

\begin{table}[H]
\centering
\caption{Feature list derived from the preliminary research}
\label{tab:feature_list}
\begin{tabular}{|l|l|l|l|}
\hline
\# & Feature                                                                                                & Priority & Source                                                                      \\ \hline
1  & Reccomendation of university and courses                                                               & 1        & Students                                                                    \\ \hline
2  & List of previously approved course matches                                                             & 2        & Students, advisers                                                          \\ \hline
3  & Matching foreign courses with NTNU courses                                                             & 2        & Students                                                                    \\ \hline
4  & Application handling for course approvals                                                              & 3        & Advisers                                                                    \\ \hline
5  & Dynamic study point allocation on courses                                                              & 4        & \begin{tabular}[c]{@{}l@{}}Office of International\\ Relations\end{tabular} \\ \hline
6   & \begin{tabular}[c]{@{}l@{}}Interactive map of exchange countries \\ by number of students\end{tabular} & 4        & Students                                                                    \\ \hline
\end{tabular}
\end{table}

\subsection{The Gap}
Among the many applications utilizing the CBR methodology, none has covered the issue of finding the most suitable country, university and courses to choose when going on an exchange study. This gap in knowledge combined with the needs identified in section \ref{sec:requirements} is what this project will target through the research goals and questions (section \ref{RQ}).
