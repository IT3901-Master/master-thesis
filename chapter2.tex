\chapter{Preliminary Research} \label{chap:2}

Before developing the research questions and deciding on the specific research methods, it was important to understand the problem domain and related work done in the field. This chapter details that process and the knowledge gained through interviews, meetings and literature review. The chapter is concluded with the identified system requirements and the targeted research gap.

\section{Interviews}

Interviews and informal meetings were conducted to understand the problem domain and gain information on the current approach of applying for an exchange program from NTNU. Two advisers from the Computer Science Department at NTNU (IDI) were interviewed, and a meeting was conducted with the Office of International Relations at NTNU (OIR). These meetings were also conducted to introduce the groups to our research goals and gain valuable feedback on possible difficulties and limitations that might be encountered in the project.

\subsection{Student Advisers} 

Student advisers were interviewed to gain a better understanding of their role in the process of approving courses for an exchange program. In the context of this process, student advisers mostly work on reviewing the courses found by students and giving them possible recommendations on courses to take, so it fits the students' study plan. They also provide information to students regarding universities with partnerships and the process as a whole. Several problem areas were noted by the advisers. Firstly, a high workload in peak periods for applications makes it difficult to allow enough time to be spent on reviewing a student's courses. Secondly, a generally large number of students makes it difficult for a personal follow-up throughout the exchange program duration. Finally, the current process of comparing and granting approvals on the courses proposed by students is arduous. The advisers were highly positive to a research project that could find ways to resolve some of these issues. Features suggested by advisers included digital applications for approval of courses and follow-up of these applications, a digital maintained list of approved courses for each department, and automatic expiration of course approvals in this list based on approval date.

\subsection{The Office of International Relations}

The reason for meeting with the staff of the OIR was to gain insight into their extensive knowledge in the field of exchange programs. Another goal was to find out if advisers from different faculties worked together and cooperated. During the meeting, the OIR presented the results of a study on student motivation they conducted in the spring of 2016. Furthermore, possible limitations with the proposed project were discussed. Among these limitations, the most prominent ones were that there might be little information available for exchange locations where few students have been before, and that there are many different rules and customs at various departments at NTNU. The OIR suggested features such as support for the differences in allocation of study points for courses at different universities, and recommendations of courses and universities at locations with fewer students.

The OIR maintains the database of experiences reported by students that have been on exchange. Gaining access to this database would be highly valuable for giving the recommendations in the prototype. The database access was however found to be restricted as it included personal information. Therefore, the OIR instead granted permission to extract the data directly from the HTML sites. 

\subsection{Exchange Students}

Because one of the authors has previously been on exchange, some of the difficulties of the current process were already identified. It was, however, useful to include students from other departments and with different experiences to gain a better understanding of the problem domain. The interviews with the students focused on issues with the current process and brainstorming possible improvements to the process. Significant issues noted by the students were the concern of courses not receiving a final approval when arriving back home after an exchange period, difficulty changing courses during the exchange period, and the tedious manual process before an exchange period.

The proposed features from students were: A map of where other students have done an exchange program, combining courses from NTNU with their respective match at the foreign university and recommendation of universities and courses specific to the preference of each student. 

\section{Current Approach}

The following description is based on the current approach at the Computer Science Department (IDI) at NTNU. It may, therefore, be some differences between the different departments. The current process for choosing and approving exchange courses is mostly done manually. See Figure \ref{ntnu_process} for a short overview of the whole application process. The boxes which are colored green are the targeted problem areas of this research project. The course approval process is required both before and after participating in an exchange program.

\begin{figure}[H]
  \centering

\begin{tikzpicture}[
  level 1/.style={sibling distance=30mm},
  edge from parent/.style={->,draw},
  >=latex]

% root of the the initial tree, level 1
\node[root] {Applying for exchange}
% The first level, as children of the initial tree
  child {node[level 2] (c1) {1. Select location}}
  child {node[level 2] (c2) {2. Make application}}
  child {node[level 2] (c3) {3. During the stay}}
  child {node[level 2] (c4) {4. After the stay}};

% The second level, relatively positioned nodes
\begin{scope}[every node/.style={level 3}]
\node [below=4pt of c1, xshift=5pt] (c11) {1.1 Talk to advisor};
\node [below=4pt of c11, color=prettyGreen,text=black] (c12) {1.2 Find experiences};
\node [below=4pt of c12, color=prettyGreen,text=black] (c13) {1.3 Decide on location};
\node [below=4pt of c2, xshift=5pt,color=prettyGreen,text=black] (c21) {2.1 Pre-approve courses};
\node [below=4pt of c21] (c22) {2.2 Create exchange application};
\node [below=4pt of c22] (c23) {2.3 Upload and send application};

\node [below=4pt of c3, xshift=5pt] (c31) {3.1 Write experience report};
\node [below=4pt of c31,color=prettyGreen,text=black] (c32) {3.2 Contact advisor in case of course change};

\node [below=4pt of c4, xshift=5pt,color=prettyGreen,text=black] (c41) {4.1 Create final course approval form};
\node [below=4pt of c41,color=prettyGreen,text=black] (c42) {4.2 Send in final approval form};

\end{scope}

\end{tikzpicture}

\caption[The current exchange program application process at NTNU]{The current exchange program application process at NTNU \footnote{https://innsida.ntnu.no/utenlandsstudier}\label{ntnu_process}}
\end{figure}

Students are first required to find and choose the university where they want to do an exchange program. There is a considerable number of possible choices, and NTNU only provide information on universities with an institutional cooperation agreement. After the university is chosen, the students have to find courses at the foreign university that can be approved as a replacement for the courses originally planned at NTNU. Most of the departments do not have a system where students can enter their replacements or find the previously approved ones. One of the exceptions, however, is the Department of Information and Computer Science which utilize a digital table for entering courses. While this table in many ways make it easier for the students to select their university and courses, the data entry does not include useful features such data validations. Manually maintaining a table is also tedious work. 

Throughout the course of an exchange program, the students often have to change their courses due to unforeseen circumstances, making the original approval invalid. To gain a new approval, a student usually has to send an email to the adviser of their department to get the new courses approved. If this process takes a while, and the course selection due date is close, the student may have to decide to register for a course without approval from NTNU, risking not getting it approved at the end of the exchange program.

\section{Related Work} \label{related_work}

Mazzarol and Soutar \cite{mazzarol2002push} found that students' general motivation is influenced by the amount of information on a university and its courses. Among the several factors which were reviewed, the \enquote{knowledge and awareness} factor proved to be the most influencing one for choosing an international study location. Therefore, when researching how an IS can improve the motivation of students to participate in an exchange program, promoting knowledge and awareness is key. 

The OIR conducted a study on 464 of their exchange students in 2016 to investigate what factors could help increase the motivation for participating in an exchange program \cite{intersek_report}. Among the 464 participating students, 84\% answered that a list of previously approved exchange courses could increase the number of students that go on exchange, and 80\% answered that pre-approved course packages could increase the numbers. This shows the importance, and possible usefulness of an IS that targets these areas.  

Several approaches have been made to replace the manual process of choosing courses with an IS. One example is by using a decision support system (DSS) that advises the students on their course selection based on their program requirements and the course's prerequisites, as done at the University of Dhaka \cite{roushan2014university}. Furthermore, student course recommendation can be done intelligently by using recommendation engines and data mining techniques to give relevant results. Sherpa \cite{bramucci2012sherpa} is a system that has been made especially for the goal of providing course recommendations. Data mining techniques can also be used to predict whether a student will fail or pass on a course \cite{vialardi2009recommendation}.

Designing a recommender system requires knowledge of the relevant data to use as attributes. Cubillo et al. \cite{maria2006international} identified the different important decision making factors for international students and their grouping. The five main groups were city effect, country image effect, personal reasons, institution image and program evaluation. Recommender systems are used extensively in both research and commercial products. These systems provide several possible solutions that might be viable for the user to choose. Several published research papers have explored different approaches to recommender systems \cite{mulyana2015case}\cite{quijano2011happy}. 

CBR is today both a recognized and well-established method in several fields of sciences; such as health science \cite{begum2011case}. Because CBR is proven to be a suitable methodology for complicated problems where uncertainty is involved \cite{richter2013case}, it has been applied to problems such as diagnosing chronic diseases in combination with data mining \cite{huang2007integrating}, and finding the most suitable study program at a senior high school \cite{mulyana2015case}. 

Case-based reasoning recommender Systems (CBR-RS) are also commonly used in decision problems, typically when there is not one exact answer, but rather several different good answers. For instance was such a system was developed to find the most suitable Massive Open Online Courses for e-learning \cite{bousbahi2015mooc}.

\section{Conclusion}

This section presents the system requirements for the prototype that were identified in the preliminary research as possible motivational factors.

\subsection{System Requirements}\label{sec:requirements}

The different needs and contributions identified in the preliminary research through literature review, interviews and analyzing the current approach are shown in the requirements list, Table \ref{tab:feature_list}. In the survey sent out by the OIR to students that have been on exchange programs, $85,34\%$ of 464 students stated that a list containing earlier approved course matches would increase the number of students going on study exchange programs \cite{intersek_report}. This functionality, therefore, felt essential to include as requirement \#2. The findings were also used to help define the research goals and questions (section \ref{RQ}). 

\begin{table}[h]
\centering
\caption{Requirement list derived from the preliminary research}
\label{tab:feature_list}
\begin{tabulary}{\textwidth}{|L|L|L|L|}
\hline
\textbf{\#} & \textbf{Requirement}                                                                                                & \textbf{Priority} & \textbf{Source}                                                                     \\ \hline \hline
1  & Recommend universities and courses                                                               & 1        & Students, OIR                                                                    \\ \hline
2  & List previously approved course matches                                                             & 2        & Students, advisers                                                          \\ \hline
3  & \begin{tabular}[c]{@{}l@{}} Support matching of exchange courses \\ with NTNU courses \end{tabular}                                                             & 2        & Students                                                                    \\ \hline
4  & Handle course approval applications                                                              & 3        & Advisers                                                                    \\ \hline
5   & \begin{tabular}[c]{@{}l@{}}Show number of exchange students in \\ each country with an interactive map\end{tabular} & 4        & Students                                                                    \\ \hline
\end{tabulary}
\end{table}

\subsection{The Gap}

Among the many applications utilizing the CBR methodology, none has covered the issue of finding the most suitable country, university and courses to choose when going on an exchange study. This gap in knowledge combined with the needs identified in section \ref{sec:requirements} is what this project will target through the research goals and questions (section \ref{RQ}).  


