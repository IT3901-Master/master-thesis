\clearpage
\pagenumbering{roman} 				
\setcounter{page}{1}

\pagestyle{fancy}
\fancyhf{}
\renewcommand{\chaptermark}[1]{\markboth{\chaptername\ \thechapter.\ #1}{}}
\renewcommand{\sectionmark}[1]{\markright{\thesection\ #1}}
\renewcommand{\headrulewidth}{0.1ex}
\renewcommand{\footrulewidth}{0.1ex}
\fancyfoot[LE,RO]{\thepage}
\fancypagestyle{plain}{\fancyhf{}\fancyfoot[LE,RO]{\thepage}\renewcommand{\headrulewidth}{0ex}}


\pagestyle{empty}
\begin{center}
\section*{\Huge\textit{Abstract}}
\end{center}



%\addcontentsline{toc}{chapter}{Summary}	
$\\[0.5cm]$


This thesis is the result of a research with the goal of investigating the potential of Case-Based Reasoning for recommending relevant universities and courses for students applying for a study abroad- or student exchange program, and how an information system utilizing this methodology can affect students' motivation for applying for such a program. By developing a system in an iterative manner with the users in focus, using the experiences of past exchange students and adapting a Case-Based Recommender System for this domain, this research seeks to both give good recommendations for students, and increase their motivation by making their application process for an exchange program easier. Results were collected by having a sample size of all students at NTNU who are interested or somehow connected to exchange studies test the developed system and answer an accompanying questionnaire. These results indicate that the proposed system would likely increase the number of students who choose to apply for a study abroad- or student exchange program, and indicates that the majority of students receives relevant recommendations for both universities and courses to choose.


\clearpage