\clearpage
\pagenumbering{roman} 				
\setcounter{page}{1}

\pagestyle{fancy}
\fancyhf{}
\renewcommand{\chaptermark}[1]{\markboth{\chaptername\ \thechapter.\ #1}{}}
\renewcommand{\sectionmark}[1]{\markright{\thesection\ #1}}
\renewcommand{\headrulewidth}{0.1ex}
\renewcommand{\footrulewidth}{0.1ex}
\fancyfoot[LE,RO]{\thepage}
\fancypagestyle{plain}{\fancyhf{}\fancyfoot[LE,RO]{\thepage}\renewcommand{\headrulewidth}{0ex}}


\pagestyle{empty}
\begin{center}
\section*{\Huge\textit{Abstract}}
\end{center}



%\addcontentsline{toc}{chapter}{Summary}	
$\\[0.5cm]$

NTNU has an objective to be internationally outstanding. One of the underlying initiatives is to increase the international mobility of NTNU's degree students. The increase can be achieved by motivating more students to go on a study abroad- or student exchange programs. This thesis introduces a new prototype created to improve the motivation for these programs by recommending courses and universities and simplifying the course approval process for students. The prototype, named \textit{Utsida}, uses case-based reasoning (CBR) and the experience reports of previous exchange students to give the recommendations. The development of the prototype was done in an iterative manner through the Design and Creation research strategy and designed in a user-centred approach. Utsida consists of two parts where each targets one of the research questions. A web application part for user interaction and data storage, and a Case-Based Reasoning Recommender System (CBR-RS) part that produces the recommendations. Utsida was evaluated by using two methods; The first being a user study were students at NTNU were able to test the Utsida prototype and answer an accompanying questionnaire. The second method tested the CBRS part in an offline experiment with simulated user queries. The user study showed that Utsida has a very positive effect on motivation for study abroad- or student exchange programs with an average agreement on statements regarding an increase in motivation of 86\%. Furthermore, the majority of students also received relevant recommendations for both universities and courses. The final results of the offline experiments were statistically significant and supported the results of the user study.

\clearpage