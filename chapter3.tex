%===================================== CHAP 3 =================================

\chapter{Research Method}\label{chap:3}

\iffalse
The developed application is at heart a mean to answer our main research question; How can intelligent recommendation increase student' motivation for studying abroad? And what information and features are important for an IS to be used? This chapter elaborates on the methods and data collection used to collect data to answer the research questions.

Using methods proven to be influential and working is essential for this project. The project does not develop any new methods but takes use of others that are well known in the research community. Such as quantitative data gathering through surveys.The methods introduced in this chapter give descriptions on how the data was gathered and the different experiments performed.

\section{Survey and questionnaires}

The surveys will be conducted through online service providers that provide the framework for the survey. The answer to the survey will then be stored in an datasheet to be analysed at a later point. Some survey sites also includes simple data analysis that give more information on the data collected in a survey.

Measuring the quality of the project as a whole will be done by analysing several factors of the research and development. We will in the project analyse the qualitative data resulting from interviews and surveys on both the manifest level and latent level of analysis. This will give a better understanding on both the clear results and what we can infer or imply from the questions.

We can use the qualitative content analysis methods presented by S. Elo and H. Kyngäs\cite{elo2008qualitative} for systematic, rule-guided qualitative text analysis in the interview and survey questions. This model will help preserve the advantages of quantitative analysis in the qualitative based research results. The steps for the content analysis is described in figure\ref{qualitytative_content}. 

The research in information systems varies to other fields by having a more practical approach to the research. The results are often given using empirical methods on users by doing usability testing and surveys. Knowing how to do perform the analysis in a correct way is important both for the quality of the results and the research validity. A book on empirical research methods have identified the common pitfalls and describes how to avoid them \cite{kitchenham2002preliminary}.

There is several relevant books and literature on analysis that should be read before starting evaluating the project. One of these include “Qualitative evaluation and research methods” by M. Q. Patton [4]. This book introduces method for analysis and evaluation that might prove useful in the project.

\begin{figure}
    \centering
    \includegraphics[width=\textwidth]{fig/method.png}
    \label{fig:qualitytative_content}
    \caption{The outlined research process\cite{oates2005researching}}
\end{figure}

\section{Development and prototyping}

\subsection{Iterative design process}
The development of the systems will be done in an iterative process to be able to quickly adapt to user feedback and changes. This will also allow constant change and tweaking of similarity functions and their weighting to better provide a relevant result.

\subsection{Development methodologies}


\section{Observation and experiments}


\subsection{Usability testing}
Usability testing will be performed during the project to evaluate system design and relevance. We will follow the guide *** to usability testing. The project will have different phases of usability testing that focus on different aspects of the system. According to *** this improves ****...???. 

\subsection{Testing Utsida on Students}
The main motivation for the research project is to make the activity of finding a university, as well as what kind of subjects is reasonable to take for a student. Because the current system used at NTNU is mainly performed manually, the process is not very seamless and straightforward. Therefore, testing this system on students throughout its development is very important.

\fi






Using Oates'\cite{oates2005researching} model of the research process, the utilized research process can be modelled as figure \ref{fig:research_process} implies.

\begin{figure}[H]
    \centering
    \includegraphics[width=0.8\textwidth]{fig/research_process.png}
    \caption{The outlined research process}
    \label{fig:research_process}
\end{figure}

Based on the motivation for doing this research, as well as the conducted literature review, the set of research questions were formed. These questions will be answered with the following strategy.

\section{Research Strategy}

This research requires the design and creation of a computer based artefact (Utsida) as a tool to answer the described research questions. However, the research strategy which will yield the desired results is considered a survey. Thus, the designated research strategy will be a combination of design and creation, and survey. 

\subsection{Design and Creation}

Utsida is the platform which users meet, as well as the required tool to answer RQ2, and partly RQ1. It will incorporate the AI methodology CBR to create a new way of recommending relevant universities and courses for a student applying for an exchange program, and serve as a united place for required and handy data and information. 

It is of great importance that this system is well functioning and easy to use, if RQ2 is to be answered. Therefore, the the design and creation research strategy has to be followed. While the system is being designed and created, it will be tested by students in an iterative manner.

\subsubsection{Usability Testing}

To ensure sufficient usability and functionality for Utsida, there will be conducted iterative sessions of usability testing on carefully selected participants, mainly students who have a decent understanding of the current system for applying for an exchange program. Both technical competent- and less technical competent students will be tested.

These tests will help finding flaws in both general functionality and design, so that the system is as reliable as possible before finally conducting remote tests. Data will be gathered by interviewing the testers, and have them fill out a System Usability Scale (SUS)-schema\cite{brooke1996sus}, and will be carefully analyzed after each test.





\subsection{Survey}
To answer RQ2, and partly RQ1, a survey will be conducted. This survey will ask students to try out Utsida, followed by a questionnaire which will establish whether the developed system can have a substantial positive effect on their motivation of applying for a study exchange, if the system would make the process easier for students who have been on a study exchange, and if they received relevant recommendations on universities and courses to choose. The feedback on the system's recommendations will contribute to RQ1, whilst the other questions will contribute to RQ2. 

The survey strategy will also be used to gain information on how to properly model the CBR module of Utsida. Each attribute in the model has to be weighted with proper importance, which require the average opinion of a large number of students to get as correct as possible. In this questionnaire, the students will be asked what they think are the most important aspects when selecting a geographical location, a university, and choosing courses for their exchange program.


\section{Data Generation Methods}

\subsection{Questionnaires}

All questionnaires which will be sent out in this research will be self-administrated, which promotes the opportunity to collect data from a large number of students at once. This method is also very cheap, and does not require as much time and planning as an administrated test would need. Because a student's thoughts of what is important, easy or convenient is rather subjective, as well as their attitude to the current system for applying for an exchange program, it is important strive to get feedback from as many students as possible, but considering that there is not much incentive for students to answer the questionnaires, and that the main survey will be rather large, a realistic goal of 30 answers is set. Considering such an amount of answers, qualitative analysis of the data will contribute more than qualitative analysis for both research questions. 

The recipients of the questionnaire should be students who have some experience with exchange studies. To acquire a larger group target group who fits this criteria, it is therefore desired to arrange a collaboration with NTNU's International Section, which regularly sends out surveys with to these students for various reasons.


\subsection{Interviews and Observation}

As a mean to acquire the needed information from the test subjects whom performed the usability tests, both observation and interviews will be done. Each test will be thoroughly observed, to see how the test subjects thought, and how they used the system, while each test will end with an interview. These data generation methods will be used to improve the system to an acceptable state before performing remote testing and questionnaires.

\subsection{Offline testing of recommendation}

To answer RQ1 and to either confirm or reject the hypothesis that CBR is a feasible methodology to use in this problem domain, an offline experiment will be conducted. This experiment is in addition to the larger online testing on users. 

The offline test focuses on comparing the tweaked CBR model with a default CBR model that is not configured to the domain. A query will be given to both the tweaked and default models and the results analyzed according to \ref{offline_test}.

Offline testing provides several benefits and some difficulties. 

\begin{table}[]
\centering
\resizebox{\textwidth}{!}{%
\begin{tabular}{|
>{\columncolor[HTML]{D0E0E3}}l |l|l|l|}
\hline
\multicolumn{4}{|c|}{\cellcolor[HTML]{A4C2F4}{\color[HTML]{333333} Rating scale for recommendations (0-10 points)}}                                                                                             \\ \hline
Attribute           & \cellcolor[HTML]{D0E0E3}0 points                                         & \cellcolor[HTML]{D0E0E3}1 point                                             & \cellcolor[HTML]{D0E0E3}2 points \\ \hline
Institute           & \begin{tabular}[c]{@{}l@{}}Wrong faculty \\ wrong institute\end{tabular} & \begin{tabular}[c]{@{}l@{}}Correct faculty\\ wrong institute\end{tabular}   & Correct Institute                \\ \hline
Year                & Before 2011                                                              & 2011-2013                                                                   & 2014-2017                        \\ \hline
Geographic Location & \begin{tabular}[c]{@{}l@{}}Wrong country \\ wrong continent\end{tabular} & \begin{tabular}[c]{@{}l@{}}Correct continent\\ wrong country\end{tabular}   & Correct country                  \\ \hline
University          & \multicolumn{2}{c|}{No match}                                                                                                                          & Perfect match                    \\ \hline
Language            & No match                                                                 & \begin{tabular}[c]{@{}l@{}}The language is \\ part of the list\end{tabular} & Perfect match                    \\ \hline
Ratings             & \multicolumn{3}{c|}{0.5 points for each rating in range (-1, rating +1)}                                                                                                                  \\ \hline
\end{tabular}%
}
\caption{Offline test score table}
\label{offline_test}
\end{table}


If the configured model yields substantially more desired results, both in terms of correct retrieval, and the most relevance for the user according to the defined rules, it is considered a success.

\section{Data Analysis}

The questionnaires which will be conducted will produce qualitative. Most of the questions in both questionnaires will be scale-based, or provide multiple predefined answers. This makes it easy to use statistical analysis of them, whilst the questions where the students can answer freely will be manually analyzed. 

\section{Ethical Issues}

\section{Practical Issues}
This research is performed by two students, with one advisor to guide the project. There is a very limited amount of monetary funds, and the time at hand is only 9 months. The data which will serve as the results will be produced by users. With the minimal budget, a practical issue will be to produce the incentive for enough students to answer the surveys. Furthermore, because of the limited time, larger administrated tests are considered out of reach. 

\cleardoublepage